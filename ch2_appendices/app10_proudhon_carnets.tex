Proudhon kept meticulous records of every book, pamphlet, or newspaper article he read over the course of most of his life, typically along with excerpts and brief summaries of these works. Thus, with these records (digitized, though still in image format, thus requiring readers to parse Proudhon's often opaque handwriting), we can approximate whether it was Marx or Proudhon who first introduced certain ideas into the lexicon of European socialist thought. Proudhon's \textit{carnets} are viewable in the form of raw
manuscript pages coded NAF 18255 to NAF 18262 at the Bibliothèque Nationale de France, and digital scans of
each page are available on the BnF’s website, at \href{https://archivesetmanuscrits.bnf.fr/ark:/12148/cc7339n}{https://archivesetmanuscrits.bnf.fr/ark:/12148/cc7339n}.

\subsubsection*{Proudhon \textit{Carnets} Sample}

As an example of the contents of these \textit{carnets}, here we reproduce the content indices for the 7 notebooks he kept in the year 1840 (\textit{Cahiers} XI--XVII):

\subsubsection*{Cahier XI, Jan 1840}
\begin{enumerate}
    \item[1-3] Réflexions prélim. sur le droit de propriété
    \item[3-4] Pellegrino Rossi, \textit{Cours d'Écon}.
	\item[4-7] Jean-Jacques Rousseau, \textit{Origines de l'Inégalité}
	\item[8-12] Jean-Baptiste Say, \textit{Écon. polit.}
	\item[12-14] On Pellegrino Rossi and Jean-Jacques Rousseau
	\item[15-16] On Rossi Pinheiro-Ferreira	
	\item[16-18] Victor Cousin, \textit{Philos. de l'hist.}
	\item[18-24] Gustave Flourens, \textit{Notice sur F. Cuvier}
    \item[24] Pellegrino Rossi, \textit{Révolution}
    \item[24-34] Ancillon, \textit{Mélanges}
    \item[34-35] Pellegrino Rossi, \textit{Propriété, force, mariage, femmes}
	\item[35-37] Ancillon, \textit{Mélanges}	
	\item[37] Anne Robert Jacques Turgot, \textit{Tom. 2 des Oeuvres complètes. Phil. de l'hist.}
    \item[38-39] Portets, \textit{Cours de droit naturel}
	\item[39-45] Barchou de Penhoen, \textit{Hist. de la phil. all.} (Bonaparte)
    \item[45-64] Victor Cousin, \textit{Hist. de la phil.} (1826) (beaucoup de choses pour mon livre)
\end{enumerate}

\subsubsection*{Cahier XII, Jan 1840}
\begin{enumerate}
    \item[1-3] F. M. A. Voltaire, \textit{Lettre de Ch. Gouju}
    \item[3-4] Marie Jean Antoine Nicolas Caritat, marquis de Condorcet, \textit{Progrès de l'espirit humain}
    \item[4] On Property
    \item[7-12] Edgar Quinet, \textit{Idées des hébreux} (idées sur la cosmogonie, de la mort, âme, etc.)
    \item[12-18] Jules Michelet, \textit{Introduction à l'hist. univ.}
    \item[18-19] \textit{Moniteur} de 1830, Révision de la charte
	\item[20-23] Conférences tenues au Conseil d'État, pour la discussion du Code civil
	\item[24-26] On Succession, marriage
    \item[26-31] Bergier, \textit{Dict. Théol.} et notes de Bousset, \textit{Usure}
    \item[31-35] J.-B. Bossuet, \textit{Usure}
	\item[35] Immanuel Kant, \textit{Principes métaphysiques de morale}
    \item[42] Réflexions sur moi-même, 7 février 1840
    \item[42] Michel Chevalier, \textit{Lettres sur l'Amérique du Nord}
    \item[43] Le \textit{National} du 4 février, on Misère, vagabondage, Lamennais, Victoria, Thiers
    \item[44-52] Pothier, \textit{Droit de propriété}
    \item[53] Toullier, \textit{Tom V. De donation et testament}
    \item[54] \textit{ibid.}, \textit{Tom. VI. De l'objet et de la matière des contrats}
    \end{enumerate}
\subsubsection*{Cahier XIII, Feb 1840}
\begin{enumerate}
    \item[1] Victor Cousin, \textit{Cours} de 1828. Grand homme
    \item[3-7] Edgar Quinet, \textit{Idées de Herder} (ses moeurs primitives)
    \item[8-12] Buckland, \textit{Géologie, couronné par l'Institut.} (Prière de Kepler)
    \item[13-20] Toullier, \textit{Prescription, présomption}, Dunod, \textit{ibid.}
    \item[20-23] Charron, \textit{Ambition, peuple}
    \item[23-25] Burnouf, Introduct. à la traduction de Tacite
    \item[25-27] Proudhon, \textit{Traité des droits d'usufruit}
    \item[27-32] Lélut, \textit{Qu'est-ce que la phrénologie?}
    \item[32-52] Garat, \textit{Mémoires sur la vie de Suard}
    \item[53-56] Joseph F. X. Droz, \textit{Art d'être heureux} (1811)
    \item[56-64] Beccaria, \textit{Traité des débats et des peines}. Faire un syllogisme parfait
\end{enumerate}

\subsubsection*{Cahier XIV, Nov. 1840}
\begin{enumerate}
    \item[1-17] Beccaria (suite) \textit{Illégitimité de la peine} 
    \item[17-28] La Luzerne (Cardinal de), \textit{Dissertation sur le prêt de commerce} (Pères cités)
    \item[29-44] Pastoret, \textit{Législation de Moïse} (et autres)
    \item[45-60] C. G. Jouffroy, \textit{Cours de Droit naturel} (âme, devoir, liberté, le \textit{National}, Spinoza)
\end{enumerate}

\subsubsection*{Cahier XV, Nov. 1840}
\begin{enumerate}
    \item[1-39] Proudhon, Leçons de Blanqui, Wolowski, citations diverses et pensées journaux
    \item[35-36] Germain Garnier, \textit{De la propriété}
    \item[40-43] \textit{Moniteur}. 20 juin 1829, Galotti (extradition de)
    \item[43-50] \textit{ibid.}, 21 juin-7 octobre, Liberté individuelle (proposition Roger sur la)	\item[50-53] François Pierre Guillaume Guizot, \textit{De la peine de mort}
    \item[53-56] Ortolan, \textit{Le ministère public}
\end{enumerate}
\subsubsection*{Cahier XVI, Nov. 1840}
\begin{enumerate}
    \item[1-5] C. G. Jouffroy, \textit{Cours de droit naturel}
    \item[6-52] P. J. B. Buchez, \textit{Philosophie} (3 vol.)
\end{enumerate}
\subsubsection*{Cahier XVII, Dec. 1840}
\begin{enumerate}
    \item[1-10] P. J. B. Buchez, \textit{Philosophie} (suite)
    \item[11-14] Troplong, \textit{Prescription}	
    \item[15-18] P. J. B. Buchez, \textit{Philosophie} (suite)
    \item[18-25] Troplong, \textit{Prescription} (suite)
    \item[25-28] Lerminier, \textit{Histoire législ. comp.}
    \item[28-35] Lerminier, \textit{Lettre phil. Phil. du droit.}
    \item[35-56] Laboulaye, \textit{Hist. du droit}. Duel, femmes, religion, etc... Délits, peines
\end{enumerate}

