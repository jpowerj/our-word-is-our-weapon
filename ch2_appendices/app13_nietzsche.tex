The books Nietzsche read, and those he borrowed or owned but probably never read, have been meticulously traced in \cite{brobjer_nietzsche_2010}, and listed in an Appendix from pages 185 to 258. We use this data source to observe that, although Nietzsche never read Marx or Engels directly (according to Brobjer's research), he \textit{was} deeply influenced by Marx and Engels' rival Eugen Dühring, discussed above. Indeed, 13 different readings of Dühring's texts are recorded in Nietzsche's reading logs: 7 readings in 1875, one each in 1878, 1881, 1883, and 1884, and then two in 1885. As a sample of the full reading logs, his readings for just the year 1875 (including the 7 Dühring texts) are reproduced here:

\begin{enumerate}
    \item E. Windisch, \textit{Indian Philosophy}
    \item Heinrich Brockhaus, \textit{Rectoratsrede über indische Philologie} (Nietzsche Attended Brockhaus's lecture ``Overview of the Results of Indian Philology'' this year)
    \item Montaigne, \textit{Essays}
    \item John William Draper, \textit{Geschichte der geistigen Entwicklung Europas} (1871)
    \item E. Windisch, (on Indian philosophy)
    \item P. Deussen, (on Indian philosophy)
    \item Eduard von Hartmann, (unknown text)
    \item G. Grote, on Plato
    \item Carl Fuchs, Unpublished notes and essays
    \item Paul Rée, \textit{Psychologische Beobachtungen} (1875) (Nietzsche writes to Rée on October 22, 1875, commenting on the book. This marks the beginning of their friendship. Nietzsche's copy of the book has a personal dedication from Rée.)
    \item E. Windisch, \textit{Tripitaka der Buddhisten}
    \item Wolfgang Senfft, \textit{Indischer Sprüche}, 3 vols.
    \item \textit{Sutta Nipáta} (in English translation)
    \item Otto von Böhtlingk, \textit{Indische Sprüche: Sanskrit und Deutsch}, 3 vols. (1870-1873)
    \item Aristotle, (unknown work)
    \item A. Schopenhauer, (unknown work)
    \item A. Schopenhauer, \textit{Die Welt als Wille und Vorstellung}
    \item Xenophon von Athen, \textit{Memorabilien}
    \item A. Schopenhauer, \textit{Parerga, Vol. 1}
    \item G. C. Lichtenberg, \textit{Einige Lebensumstände von Capt. James Cook, grössentheils aus schriftlichen Nachrichten einiger seiner Bekannten gezogen in Vermischte Schriften} (1867)
    \item F. M. A. Voltaire, (unknown work)
    \item Eugen Dühring, \textit{Der Werth des Lebens: Eine philosophische Betrachtung} (Breslau, 1865) (Nietzsche made a 63-page summary with comments.
    N acquired this book May 26, 1875.
    \item Eugen Dühring, \textit{Cursus der Philosophie als streng wissenschaftlichen Weltanschaaung und Lebensgestaltung} (Leipzig, 1875) (Nietzsche acquired this book on April 21, 1875, and read it in the summer of 1875. He later reread it in 1885.)
    \item K. Hillebrand, ``Schopenhauer und das deutsche Publikum,'' in \textit{Zeiten, Völker und Menschen, Vol. 2} (Berlin, 1875)
    \item A. Schopenhauer, \textit{Sämtliche Werke, Frauenstädt}, 6 vols. (Leipzig, 1873-1874) (Nietzsche bought this book in 1875.)
    \item Eugen Dühring, \textit{Kritische Geschichte der Philosophie von ihren Anfängen bis zur Gegenwart} (Berlin, 1873) (Nietzsche bought this book in 1875.)
    \item W. Oncken, \textit{Die Staatslehre des Aristoteles in historisch-politischen Umrissen}, 2 vols.	(Leipzig, 1870) (Nietzsche bought this book in 1875.)
    \item Eugen Dühring, \textit{Natürliche Dialektik} (Berlin, 1865) (Nietzsche bought this book in 1875.)
    \item Eugen Dühring, \textit{Kritische Geschichte der allgemeinen Prinzipien der Mechanik} (Berlin, 1873) (Nietzsche bought this book in 1875.)
    \item Plato, many different booklets (Nietzsche bought these in 1875.)
    \item Confucius, \textit{Ta-Hio} (Nietzsche bought this book in 1875.)
    \item Confucius, \textit{Lao-tse tao} (Nietzsche bought this book in 1875.)
    \item K. Hillebrand, \textit{Zeiten, Völker und Menschen}, 2 vols. (Berlin, 1875) (Nietzsche bought this book in 1875.)
    \item Eugen Dühring, \textit{Kursus der National- und Sozialökonomie, einschliesslich der Hauptpunkte der Finanz-politik} (Berlin, 1873) (Nietzsche bought this book in 1875.)
    \item Eugen Dühring, \textit{Kritische Geschichte der Nationalökonomie und des Sozialismus} (Berlin, 1875) (Nietzsche bought this book in 1875.)
    \item John William Draper, \textit{Geschichte der Conflicte zwischen Religion und Wissenschaft, Int. wiss. Bibl. XIII} (1875) (Nietzsche bought this book in 1875.)
    \item Herbert Spencer, \textit{Einleitung in das Studium der Sociologie, Vol. 1, Int. wiss. Bibl. XIV} (1875) (Nietzsche bought this book in 1875.)
    \item Herbert Spencer, \textit{Einleitung in das Studium der Sociologie, Vol. 2, Int. wiss. Bibl. XV} (1875) (Nietzsche bought this book in 1875.)
    \item Paul Rée, \textit{Psychologische Beobachtungen} (Berlin, 1875) (Nietzsche bought this book in 1875.)
    \item Aristotle, \textit{Rhetorik} (University teaching, summer term of 1875)
    \item Diogenes Laertius, \textit{Demokritos}	(University teaching, winter term of 1875-1876)
    \item Plato, \textit{Phaedon} (Paedagogium teaching, winter term of 1875-1876.)
    \item Plato, \textit{Protagoras} (Paedagogium teaching, winter term of 1875-1876.)
    \item Plato, \textit{Symposion} (Paedagogium teaching, winter term of 1875-1876.)
    \item Plato, \textit{Phaedrus} (Paedagogium teaching, winter term of 1875-1876.)
    \item Plato, \textit{Politeia} (Paedagogium teaching, winter term of 1875-1876.)
    \item George Henry Lewes, \textit{Geschichte der Philosophie von Thales bis Comte}, Bd. 1 (Berlin, 1871) (Borrowed from the Basel University library, 1875.)
    \item Aristotle, \textit{Ars rhetorica}, hrsg. von L. Spengel, 2 Bde. (Leipzig, 1867) (Borrowed from the Basel University library, 1875.)
    \item Democritus, \textit{Operum fragmenta}, hrsg. von F. W. A. Mullach (Berlin, 1843) (Borrowed from the Basel University library, 1875.)
    \item Staphanus Byantinus, \textit{Ethicorum quae supersunt}, hrsg. von A. Meineck, Bd. 1 (Berlin, 1849) (Borrowed from the Basel University library, 1875.)
\end{enumerate}

The full dataset is available at \href{https://airtable.com/shrkclLiN1GHJlQ2h}{https://airtable.com/shrkclLiN1GHJlQ2h}

% Password ``mega2022''