\noindent The corpus is divided into 11 different groups of texts -- 5 ``primary'' corpora analyzed to track Marx's thought over time with respect to the broader socialist movement, and 6 ``control'' corpora used for robustness tests.

\subsection{Early Marx\label{app:earlymarx}}

We define the ``early Marx'' to be the period of Marx's life before he had avowedly committed himself to the study of political economy as his central task -- namely, the period after the 1848 Revolutions had been crushed and he found himself in exile in London, at which point he obtained a membership card for the library of the London Museum and began his ``deep dive'' into political economy.

%As can be seen in Figure \ref{fig:marxpolecon}, although he did his first readings in political economy as early as 1843 (when, sparked by his excitement upon reading Engels' ``Grundwerke'' while in Paris, he immediately read all of the political-economic literature cited by Engels, in French editions).

%\begin{figure}
%    \label{fig:marxpolecon}
    %\includegraphics{marx_polecon.png}
%\end{figure}

\subsection{Late Marx\label{app:latemarx}}

Given the Early Marx/Late Marx split defined in the previous section, the following texts are examples of texts categorized under ``Late Marx'':
\begin{enumerate}
    \item \textit{Grundrisse} (1859-1862 Manuskripte)
    \item \textit{Theorien des Mehrwert} (Theories of Surplus Value) (1864-1866 Manuskripte)
    \item \textit{Kapital, Vol. I}
    \item \textit{Kapital, Vol. II}
    \item \textit{Kapital, Vol. III}
    \item \textit{Das Bürgerkrieg in Frankreich}
\end{enumerate}

\subsection{Early Socialist Texts}

Since we want to track changes in Marx's thought \textit{with respect to} the broader socialist discourse into which he aimed to intervene, we use the same split point as with the Early Marx/Late Marx texts: the downfall of the Revolutions of 1848. Under this definition, the texts classified as ``early socialist'' are as follows:

\begin{enumerate}
    \item François-Noël Babeuf (via Filippo Buonarroti)
    \item Louis Blanc, \textit{Geschichte}
    \item Louis-Auguste Blanqui
    \item J. F. Bray
    \item Etienne Cabet
    \item Théodore Dézamy
    \item Barthélemy-Prosper Enfantin
    \item Charles Fourier
    \item Moses Hess
    \item Robert Owen, \textit{New Moral World}
    \item Pierre-Joseph Proudhon
    \item Olinde Rodrigues
    \item Henri de Saint-Simon
    \item Lorenz von Stein
    \item Wilhelm Weitling
\end{enumerate}

\subsection{Late Socialist Texts}

Using the definition described in the previous section, the authors whose works are classified as ``late socialist'' are as follows:

\begin{enumerate}
    \item Mikhail Bakunin
    \item August Bebel
    \item Eduard Bernstein
    \item Karl Blind
    \item Karl Grün
    \item Alexander Herzen
    \item Friedrich Lange
    \item Ferdinand Lassalle
    \item Wilhelm Liebknecht
    \item Prosper-Olivier Lissagaray
    \item Georgei Plekhanov
    \item Johann Karl Rodbertus
    \item Konrad Schramm
    \item Karl Vogt
\end{enumerate}

\subsection{Hegelian Texts}

\begin{enumerate}
    \item Hegel
    \item Bruno Bauer
    \item Edgar Bauer
    \item August Cieszkowski
    \item Arnold Ruge
    \item Max Stirner
    \item D. F. Strauss, \textit{Das Leben Jesu}
    \item Ludwig Feuerbach
\end{enumerate}

\subsection{Political-Economic Texts}

Since these are not being tracked over time, but rather are being used to place the texts of Marx and the socialist movement on a Hegel-vs-Political-Economy spectrum, these works span the period from 1776 to 1876. The two key texts for Part I, however, are:
\begin{enumerate}
    \item David Ricardo, \textit{Principles of Political Economy and Taxation} (1819)
    \item Adam Smith, \textit{Wealth of Nations} (1776)
\end{enumerate}
A sampling of additional key political-economic authors whose works are included in the corpus:
\begin{enumerate}
    \item Frederic Bastiat
    \item Henry Carey
    \item Thomas Hodgskin
    \item William Jevons
    \item Richard Jones
    \item Thomas Joplin
    \item Friedrich List
    \item James Ramsay MacCulloch
    \item T. R. Malthus
    \item Karl Menger
    \item John Stuart Mill
    \item H. F. Osiander
    \item William Petty
    \item François Quesnay
    \item Piercy Ravenstone
    \item Jean-Baptiste Say
    \item Nassau William Senior
    \item J.-C.-L. Simonde de Sismondi
    \item James Steuart
    \item H. F. von Storch
    \item Robert Torrens
    \item François Villegardelle
\end{enumerate}

Lastly, prominent political-economic periodicals like \textit{The Economist} and \textit{Westminster Review} are included for the relevant periods.

\subsection{Classical Literature}

This and the remaining categories were created as ``control groups'', and thus span across the 17th, 18th, and 19th centuries.

\begin{enumerate}
    \item Dante
    \item Goethe, \textit{Faust}
    \item G. E. Lessing
    \item Homer
    \item Lucretius
    \item Shakespeare
\end{enumerate}

\subsection{Contemporary Literature}

\begin{enumerate}
    \item Ferdinand Freiligrath
    \item Heinrich Heine
    \item Georg Herwegh
    \item Victor Hugo
    \item Eugène Sue
\end{enumerate}

\subsection{Classical Philosophy}

This category includes, for example, the Enlightenment \textit{philosophes} who probably most influenced Marx's thought in the second-order sense, by influencing radicals of the French Revolution like Babeuf who subsequently influenced the 19th century European socialist movement. It also includes early German philosophers, mostly contemporaneous to Hegel, like Fichte and Schelling, as well as truly classical philosophers like Epicurus and Democritus (the two subjects of Marx's doctoral dissertation).

\begin{enumerate}
    \item Aristotle
    \item Cicero
    \item Democritus
    \item Antoine Destutt de Tracy
    \item Denis Diderot
    \item Epicurus
    \item J. G. Fichte
    \item J. G. Herder
    \item Baron d'Holbach
    \item David Hume
    \item Immanuel Kant
    \item John Locke
    \item Baron de Montesquieu
    \item Plato
    \item H. S. Reimarus
    \item Jean-Jacques Rousseau
    \item F. W. J. Schelling
    \item Benedict de Spinoza
\end{enumerate}

\subsection{Contemporary Philosophy}

Philosophers writing around the same time as Marx, but not as part of the socialist movement. This includes both politically-engaged writers, like Antoine-Elisée Cherbuliez and Alexis de Tocqueville, and ostensibly ``apolitical'' writers like Charles Augustin Sainte-Beuve.

\begin{enumerate}
    \item Benjamin Franklin
    \item Giuseppe Mazzini
    \item Alexis de Tocqueville
\end{enumerate}

\subsection{General History}

A large, probably under-analyzed chunk of Marx's reading was on history, especially histories of Rome and on the feudal origins of the contemporary European societies whose political and economic systems he was critiquing\footnote{Though, as argued by e.g. Kevin Anderson, towards the end of his life he began reading much more deeply on societies outside the ``core'' developed societies of Europe, especially India, Indonesia (Java), and Russia.}

\begin{enumerate}
    \item Gustav von Gülich
    \item John Lubbock
    \item Henry Sumner Maine
    \item G. L. von Maurer
    \item Lewis Henry Morgan
    \item J. B. Phear
\end{enumerate}

\subsection{Natural Sciences}

Much (probably too much) ink has been spilled on Marx's relationship to natural scientists, especially Charles Darwin. Though peripheral to our concerns in this work, this category was constructed as an additional control group, with texts such as:

\begin{enumerate}
    \item Charles Babbage, \textit{On the Economy of Machinery and Manufactures} (1832)
    \item Charles Darwin, \textit{The Origin of Species} (1859)
    \item Andrew Ure, \textit{The Philosophy of Manufactures} (1835)
\end{enumerate}

\subsection{Miscellaneous Texts}

There exist several texts which can be identified as influential on Marx's thought despite not fitting naturally into any of the previous categories: for example, the Lexicons, parliamentary Blue Books, and statistical compendia which he cited throughout his journalistic and political-economic writings.

\begin{enumerate}
    \item Lexikon
    \item Blue Books
    \item Russian statistical compendia
\end{enumerate}
