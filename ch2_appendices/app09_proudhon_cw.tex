The most frequently-referenced collection of Proudhon's works is known as the ``Rivière edition'':
\begin{quote}
    \textit{Oeuvres complètes de P.-J. Proudhon}, nouvelle édition, ed. C. Bouglé et H. Moysset (Paris: Marcel Rivière, 1923--1959), 15 vols. in 19, in-8°.
\end{quote}
This collection\footnote{Digitized versions of every volume, with the exception of the first part of Volume 1, are available on \href{https://catalog.hathitrust.org/Record/010062933}{HathiTrust}. Links to the individual volumes are given after the years of publication in the listing herein.} consists of 15 volumes published between 1923 and 1959, the contents of which are as follows:
\begin{enumerate}
    \item \textit{Système des contradictions économiques, ou philosophie de la misère}, 2 vols., 1923. \href{https://hdl.handle.net/2027/uc1.b4149255}{HathiTrust} (Part 2 only)
    
    \item \textit{Idée générale de la révolution au XIXe siècle}, 1924. \href{https://hdl.handle.net/2027/uc1.b4149256}{HathiTrust}
    
    \item \textit{De la Capacité politique des classes ouvrières}, 1924. \href{https://hdl.handle.net/2027/uc1.b4149257}{HathiTrust}
    
    \item \textit{Candidature à la pension Suard. De la célébration du Dimanche. Qu'est-ce que la propriété?}, 1926. \href{https://hdl.handle.net/2027/uc1.b4149258}{HathiTrust}
    
    \item \textit{De la création de l'ordre dans l'humanité, ou principes d'organisation politique}, 1927. \href{https://hdl.handle.net/2027/uc1.b4149259}{HathiTrust}
    
    \item \textit{La Guerre et la paix, recherches sur le principe et la constitution du droit des gens}, 1927. \href{https://hdl.handle.net/2027/uc1.b4149260}{HathiTrust}
    
    \item \textit{Les Confessions d'un révolutionnaire pour servir à l'histoire de la révolution de février}, 1929. \href{https://hdl.handle.net/2027/uc1.b4149261}{HathiTrust}
    
    \item \textit{De la Justice dans la révolution et dans l'Eglise}, 4 vols., 1930. HathiTrust: \href{https://hdl.handle.net/2027/uc1.b4149262}{Part 1}, \href{https://hdl.handle.net/2027/uc1.b4149263}{Part 2}, \href{https://hdl.handle.net/2027/uc1.b4149264}{Part 3}, \href{https://hdl.handle.net/2027/uc1.b4149265}{Part 4}
    
    \item \textit{La Révolution sociale démontrée par le coup d'état du deux décembre. Projet d'exposition perpétuelle}, 1936. \href{https://hdl.handle.net/2027/uc1.b4149266}{HathiTrust}
    
    \item \textit{Deuxième Mémoire sur la propriété. Avertissement aux propriétaires. Programme révolutionnaire. Impôt sur le revenu. Le Droit au travail et le droit de propriété}, 1938. \href{https://hdl.handle.net/2027/uc1.b4149267}{HathiTrust}
    
    \item \textit{Du Principe de l'art et sa destination sociale. Galilée. Judith. La Pornocratie ou les femmes dans les temps modernes}, 1939. \href{https://hdl.handle.net/2027/uc1.b4149268}{HathiTrust}
    
    \item \textit{Philosophie du progrès. La Justice poursuivie par l'Eglise}, 1946. \href{https://hdl.handle.net/2027/uc1.b4149269}{HathiTrust}
    
    \item \textit{Contradictions politiques. Les Démocrates assermentés et les réfractaires. Lettre aux ouvriers en vue des élections de 1864. Si les traités de 1815 ont cessé d'exister?}, 1952. \href{https://hdl.handle.net/2027/uc1.b4149270}{HathiTrust}
    
    \item \textit{Du Principe fédératif. La Fédération et l'unité en Italie. Nouvelles Observations sur l'unité italienne. France et Rhin (fragments)}, 1959. \href{https://hdl.handle.net/2027/uc1.b4149271}{HathiTrust}
    
    \item \textit{Ecrits sur la religion}, 1959. \href{https://hdl.handle.net/2027/uc1.b4149272}{HathiTrust}
\end{enumerate}

An earlier collection, which actually began publication while Proudhon was still actively writing in 1850 (with the final volume published in 1872), has also been digitized by the Bibliothèque Nationale de France. The scanned pages of all 26 volumes are available at \href{https://catalogue.bnf.fr/ark:/12148/cb31154797t}{https://catalogue.bnf.fr/ark:/12148/cb31154797t}

%The contents of the 26 volumes are as follows:

% \begin{itemize}
%     \item[] Vol I.Nouvelle édition. - 1867
%     \begin{itemize}
%         \item Qu'est-ce que la propriété ?
%         \item 1er mémoire : Recherches sur le principe du droit et du gouvernement
%         \item 2e mémoire : Lettre à M. Blanqui sur la propriété. 
%     \end{itemize}
%     \item[II.] Avertissement aux propriétaires. La célébration du dimanche, plaidoyer devant la cour d'assises de Besançon. De la concurrence entre les chemins de fer et les voies navigables. Le "Miserere". Nouvelle édition. - 1868
%     \item[III.] De la création de l'ordre dans l'humanité, ou Principes d'organisation politique. Nouvelle édition. - 1868
%     \item[IV-V.] Système des contradictions économiques, ou Philosophie de la misère. 2e édition. - 1850
%     item[VI.] Solution du problème social : Solution du problème social ; Organisation du crédit et de la circulation ; Résumé de la question sociale ; Banque d'échange ; Banque du peuple. Suivie du rapport de la Commission des délégués du Luxembourg. Nouvelle édition. - 1868
%     \item[VII.] La révolution sociale démontrée par le coup d'État du 2 décembre [précédée de la lettre de Proudhon au Prince-Président]. Le droit au travail et le droit de propriété. L'impôt sur le revenu. Nouvelle édition. - 1868
%     \item[VIII.] Du principe fédératif et de la nécessité de reconstituer le parti de la Révolution. Si les traités de 1815 ont cessé d'exister, actes du futur congrès. Nouvelle édition. - 1868
%     \item[IX.] Les confessions d'un révolutionnaire pour servir à l'histoire de la révolution de février. Nouvelle édition revue, corrigée et augmentée par l'auteur. - 1868
%     \item[X.] Idée générale de la révolution au XIXe siècle (choix d'études sur la pratique révolutionnaire et industrielle). Nouvelle édition. - 1868
%     \item[XI.] Manuel du spéculateur à la Bourse. - 1869
%     \item[XII.] Des réformes à opérer dans l'exploitation des chemins de fer... Nouvelle édition. - 1868
%     \item[XIII-XIV.] La guerre et la paix, recherches sur le principe et la constitution du droit des gens. Nouvelle édition. - 1869
%     \item[XV.] Théorie de l'impôt, question mise au concours par le conseil d'état du canton de Vaud en 1860. Nouvelle édition. - 1868
%     \item[XVI.] Les majorats littéraires, examen d'un projet de loi ayant pour but de créer au profit des auteurs, inventeurs et artistes, un monopole perpétuel. La fédération et l'unité en Italie. Nouvelles observations sur l'unité italienne. Les démocrates assermentés et les réfractaires. - 1868
%     \item[XVII-XIX.] Mélanges. Articles de journaux, 1848-1852. - 1868-1871
%     \item[XX.] La philosophie du progrès. La justice poursuivie par l'Église. Nouvelle édition. - 1868
%     \item[XXI-XXVI.] Essais d'une philosophie populaire : de la justice dans la révolution et dans l'Eglise. Nouvelle édition. - Notes et éclaircissements. - 1868-1870e
% \end{itemize}

% \begin{itemize}
%     \item Ier· — Qu’est-ce que la propriété ? (1er et 2e Mémoire). Lettre à Blanqui.
% IIe. — 1o Avertissement aux propriétaires ; 2o Plaidoyer de l’auteur devant la cour d’assises de Besançon ; 3o Célébration du dimanche ; 4o De la concurrence entre les chemins de fer et les voies navigables : 5o Le Miserere.
% IIIe. — Création de l’ordre dans l’humanité.
% IVe et Ve. — Système des contradictions économiques. Philosophie de la misère.
% VIe. — Solution du problème social. Organisation du crédit. Résumé de la question sociale. Banque d’échange, Banque du peuple.
% VIIe. — La Révolution sociale. Droit au travail et droit de propriété. — L’Impôt sur le revenu.
% VIIIe — 1o Du principe fédératif ; 2o Si les traités de 1815 ont cessé d’exister.
% IXe. — Confessions d’un révolutionnaire.
% Xe. — Idée générale de la Révolution au XIXe siècle.
% XIe. — Manuel du spéculateur à la Bourse.
% XIIe. — Des Réformes à opérer dans l’exploitation des chemins de fer.
% XIIIe et XIVe. — La Guerre et la Paix.
% XVIe. — Théorie de l’impôt.
% XVIe — 1o Majorats littéraires ; 2o Fédération et unité en Italie ; 3o Nouvelles observations sur l’unité italienne; 4o Les démocrates assermentés.
% XVIIe, XVIIIe et XIXe. — Brochures et articles de journaux depuis février 1848 jusqu’à 1852, réunis pour la première fois articles du Représentant du peuple, du Peuple, de la Voix du peuple, du Peuple de 1850, Idées révolutionnaires. Intérêt et capital.
% XXe — La justice poursuivie par l’Église, Philosophie du Progrès.
% XXIe, XXIIe, XXIIIe et XXIVe. — De la Justice dans la Révolution et dans l’Église.
% XXVe. — Mélanges divers. — Notes de la Justice.
% \end{itemize}


