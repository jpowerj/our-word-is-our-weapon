A dataset of every reference to another work across all of Marx's writings ($N \approx 10000$), available at \href{https://airtable.com/shrhgCVgr2eAbySEK}{https://airtable.com/shrhgCVgr2eAbySEK}.

%, password ``\texttt{mega2021}''.

\subsubsection*{References Sample}

As a small sample of the references dataset, here are the first 10 and last 10 (in alphabetical order) of the 140 authors referenced by Marx, as listed in \textit{Marx-Engels Werke} (\textit{MEW}), Volume 1:

\paragraph*{First 10}

\begin{enumerate}

\item Archibald Alison: \textit{The principles of population, and their connection with human happiness}. Vol. 1-2. London 1840.

\item \textit{Allgemeines Landrecht für die Preußischen Staaten}. Th. 2. Berlin 1806.

\item Aristotle: \textit{Metaphysica}.

\item Aurelius Augustinus: \textit{De civitate Dei}.

\item Francis Bacon: \textit{De dignitate et augmentis scientiarum}. T. 1. Wirceburgi
1779.

\item Bruno Bauer
\begin{enumerate}
    \item \textit{Die Fähigkeit der heutigen Juden und Christen, frei zu werden}. In: Einundzwanzig Bogen aus der Schweiz. Hrsg. von Georg Herwegh. Th. 1. Zürich und Winterthur 1843.
    \item \textit{Die Judenfrage}. Braunschweig 1843.
    \item \textit{Kritik der evangelischen Geschichte der Synoptiker}. Bd. 1-2. Leipzig 1841-1842. Bd. 3. Braunschweig 1842.
    \item \textit{Die Posaune des jüngsten Gerichts über Hegel den Atheisten und Antichristen. Ein Ultimatum}. Leipzig 1841.
\end{enumerate}

\item Edgar Bauer: \textit{Der Streit der Kritik mit Kirche und Staat}. Charlottenburg 1843.

\item Gustave de Beaumont: \textit{Marie oü l'esclavage aus Etats-Unis, tableau de moeurs americaines; l'uri des auteurs de l'öuvrage intitule: Du systeme penitentiaire aux Etats-Unis. T. 2}. Paris 1835.

\item \textit{Die Bibel}.
\begin{enumerate}
    \item \textit{Das Alte Testament}.
    \begin{enumerate}
        \item 1. Buch Mose.
        \item Buch Josua.
        \item Hosea.
    \end{enumerate}
    \item \textit{Das Neue Testament.}
    \begin{enumerate}
        \item Evangelium des Matthäus.
        \item Evangelium des Markus.
        \item Evangelium des Lukas.
        \item Evangelium des Johannes.
        \item Apostelgeschichte des Lukas.
        \item Brief des Paulus an die Römer.
        \item I. Brief des Paulus an die Korinther.
    \end{enumerate}
\end{enumerate}

\item William Blackstone: \textit{Commentaries on the laws of England. Vol. 1-4}. London 1826.

% # 11
% \item \textit{Bluntschli, Johann Kaspar}: Die Kommunisten in der Schweiz nach den bei Weitling vorgefundenen Papieren. Wörtlicher Abdruck des Kommissionalberichtes an die H. Regierung des Standes Zürich. Zürich 1843.

% \item \textit{Börne, Ludwig}: Briefe aus Paris. In: Gesammelte Schriften. Th. 9-14, Hamburg 1832, Paris 1833-1834.

% \item \textit{Buchez, Philippe-Joseph-Benjamin et Pierre-Celestin Roux}: Histoire parlementaire la Evolution française, ou Journal des Assemblées nationales depuis 1789 jusqu'en 1815. T. 28. Paris 1836.

% \item \textit{Bülow-Cummerow, Ernst Gottfried Georg von}: Preußen, seine Verfassung, seine Verwaltung,
% sein Verhältniß zu Deutschland. Berlin 1842.

% \item \textit{Buonarroti, Philippe}
% \begin{enumerate}
%     \item Conspiration pour l'égalité dite de Babeuf, suivie du procès auquel eile donna lieu, et des pièces justificatives, etc., etc. T. 1-2. Bruxelles 1828.
%     \item History of Babeuf's conspiracy for equality; with the author's reflections on the causes and character of the French Revolution, and his estimate of the leading men and events of that epoch. Also, his views of democratic government, community of property, and political and social equality. London 1836.
% \end{enumerate}

% \item \textit{Cabet, Etienne}: Voyage en Icarie, roman philosophique et social. 2. éd. Paris 1842.

% \item \textit{Carlyle, Thomas}
% \begin{enumerate}
%     \item Chartism. London 1840.
%     \item Past and present. London [1843].
% \end{enumerate}
    
% \item \textit{Carrière, Moriz}: Gedichte von Ferdinand Freiligrath, Stuttgart und Tübingen, 1838. In: Jahrbücher für wissenschaftliche Kritik, Nr. 8, Januar 1839.

% \item \textit{Cervantes Saavedra, Miguel de}: Don Quijote.

% \item \textit{La Charte Constitutionelle}.

% \item \textit{Chevalier, Michel}: Des intérêts matériels en France. Travaux publics. Routes. Canaux. Chemins
% de fer. Paris und Bruxelles 1838.

% \item \textit{Cirkular-Verfügung an sämmtliche Königl. Oberpräsidien, die Handhabung der Censur betreffend, vom 24. Dezember 1841}. In: Ministerial-Blatt für die gesammte innere Verwaltung in den Königlich Preußischen Staaten. Berlin. 2. Jg., Nr. 15, 27. Dezember 1841.

% \item \textit{Clausen, Johann Christoph Heinrich}: Pindaros der Lyriker. In: Programm womit zu der öffentlichen Prüfung der Zöglinge des Gymnasiums zu Elberfeld welche den 15. und 16. September 1834, Vormittags von 8 und Nachmittags von 2 Uhr an, in dem Gymnasial-Gebäude abgehalten werden soll, sowie zu dem Rede-Act und der Abiturienten-Entlassung am 16. d. M. Nachmittags 2 Uhr einen löblichen Schulvorstand, sämtliche Eltern der Schüler, deßgleichen alle Freunde und Gönner des höheren Schulwesens überhaupt und der Anstalt insbesondere im Namen des Lehrer-Collegiums ehrerbietigst einladet. Elberfeld 1834.

% \item \textit{[Coblenz, Peter]}:
% \begin{enumerate}
%     \item Antheil der Moselbewohner an der ferneren Bewegung der Presse. In: Rheinische Zeitung, vom 12.Dezember 1842.
%     \item Ueber die nothwendige Freigebung des Gemeindeeigenthums. In: Rheinische Zeitung, vom
%     14. Dezember 1842.
% \end{enumerate}

% \item \textit{Constant, Benjamin}: De la religion, considérée dans sa source, ses formes et ses développements. T. 1. 2.éd. Paris 1826.

% \item \textit{Constitution de la république française, décrétée par la Convention nationale et acceptée par le peuple dans le mois de fructidor an 3, promulguée le 1er vendémiaire an 3. (Aôut et septembre 1795.)} Déclaration des droits et des devoirs de l'homme et du citoyen. In: Philippe-Joseph-Benjamin Buchez et Pierre-Célestin Roux: Histoire parlementaire de la révolution
% française, ou Journal des Assemblées nationales depuis 1789 jusqu'en 1815. T. 35. Paris 1837.

% \item \textit{Constitution de 1793. Mise en discussion le 11 juin 1793; -- Achevée le 24 du même mois.} Declaration des droits de l'homme et du citoyen. In: Philippe-Joseph-Benjamin Buchez et Pierre-Célestin Roux: Histoire parlementaire de la révolution française, ou Journal des Assemblées nationales depuis 1789 jusqu'en 1815. T. 31. Paris 1837. 362-367

% \item \textit{Constitution française. Décrétée par l'assemblée nationale Constituante aux années 1789, 1790 et 1791}. Declaration des droits de l'homme et du citoyen. In: Philippe-Joseph-Benjamin Buchez et Pierre-Célestin Roux: Histoire parlementaire de la révolution française, ou Journal des Assemblées nationales depuis 1789 jusqu'en 1815. T. 11. Paris 1834.

% \item \textit{Cousin, Victor}: Über französische und deutsche Philosophie. Stuttgart und Tübingen 1834.

% \item \textit{[Delolme, Jean-Louis]}: Constitution de l'Angleterre. Amsterdam 1771.

% \item \textit{Dingelstedt, Franz}: Ferdinand Freiligrath. Ein Literaturbild. In: Jahrbuch der Literatur. 1.-Jg. Hamburg 1839.

% \item \textit{Duns Scotus, Johannes}: Opus oxoniense sive anglicanum.

% \item \textit{Eichhoff, Karl, und Karl Christian Beltz}: Lateinische Schulgrammatik mit Rücksicht auf die neuere Gestaltung der deutschen Sprachlehre, für die unteren und mittleren Gymnasialklassen und für Progymnasien. Elberfeld 1837.

% \item \textit{Einundzwanzig Bogen aus der Schweiz}. Hrsg. von Georg Herwegh. Zürich und Winterthur 1843.

% \item \textit{Erneuertes Censur-Edict für die Preußischen Staaten exclusive Schlesien}. Berlin, den 19. December 1788. [Berlin o.J.]

% \item \textit{Ewich, Johann Jacob}: Human, der Lehrer einer niederen und höheren Volksschule in seinem
% Wesen und Wirken. Th. 1-2. Wesel 1829.

% \item \textit{Feuerbach, Ludwig}
% \begin{enumerate}
%     \item Vorläufige Thesen zur Reformation der Philosophie. In: Anekdota zur neuesten deutschen Philosophie und Publicistik. Bd. 2. Zürich und Winterthur 1843.
%     \item Feuerbach, Ludwig: Das Wesen des Christenthums. Leipzig 1841.
% \end{enumerate}

% \item \textit{Fonblanque, Albany}: England under seven administrations. Vol. 1--3. London 1837.

% \item \textit{Freiligrath, Ferdinand}
% \begin{enumerate}
%     \item Der ausgewanderte Dichter. Weitere Bruchstücke, Fünf Gedichte ohne einzelne Überschriften. In: Morgenblatt für gebildete Leser, vom 10. September 1836.
%     \item Meerfahrt.
%     \item Schwalbenmärchen.
% \end{enumerate}

% \item \textit{[Friedrich Wilhelm III]}: [An den Ober-Präsidenten von Bodelschwingh-Velmede. Berlin,
% 3. Juli 1836.] In: Amts-Blatt der Königl. Regierung zu Coblenz, vom 21. Juli 1836.

% \item \textit{[Giehne, Friedrich Wilhelm]}: Der Zollcongreß. Karlsruhe, 8. Oct. In: Allgemeine Zeitung, vom 11. Oktober 1842.

% \item \textit{Godwin, William}
% \begin{enumerate}
% \item Enquiry concerning political justice, and its influence on general virtue and happiness. Vol. 1. London 1793.

% \item An enquiry concerning political justice, and its influence on morals and happiness. Vol. 2.
% London 1793.
% \end{enumerate}

% \item \textit{Görres, Joseph von}: Kirche und Staat nach Ablauf der Cölner Irrung. Weißenburg 1842.

% \item \textit{Goethe, Johann Wolfgang von}
% \begin{enumerate}
%     \item Falconet und über Falconet.
%     \item Iphigenie auf Tauris.
%     \item Rechenschaft.
%     \item Reineke Fuchs.
% \end{enumerate}

% \item \textit{Graham, James Robert George}: Factories' education. In: Hansard's Parliamentary Debates. Debates: Third Series; commencing with the accession of William IV. Vol. 67. Comprising the period from the twenty-eighth day of February, to the twenty-fourth day of March, 1843. London 1843.

% \item \textit{Gregor XVI}: Hirtenbrief Seiner Päpstlichen Heiligkeit Gregor XVI. an alle Bischöfe der katholischen Welt. Oder das Urtheil der Kirche Christi über den Geist, die Richtungen und Gefahren dieser Zeit. Erlassen in Rom den 15. August 1832. Orig. u. dt. Uebers. Regensburg [1832].

% \item \textit{Güll, Friedrich}: Kinderheimath in Bildern und Liedern. Mit einem Vorwort von Gustav Schwab. Stuttgart 1837.

% \item \textit{Gutzkow, Karl}
% \begin{enumerate}
%     \item Patkul. In: Dramatische Werke. Bd. 2. Leipzig 1842.
%     \item Werner, oder Herz und Welt. In: Dramatische Werke. Bd. 1. Leipzig 1842.
% \end{enumerate}

% \item \textit{Haase, Friedrich}: Lateinische Grammatik. In: Ergänzungsblätter zur Allgemeinen Literatur-Zeitung. Nr. 65-70, August 1838.

% \item \textit{Haller, Carl Ludwig von}: Restauration der Staats-Wissenschaft oder Theorie des natürlichgeselligen Zustands, der Chimäre des künstlich-bürgerlichen entgegengesezt. Bd. 1-6. Winterthur 1816-1834.

% \item \textit{Hals oder Peinliche Gerichtsordnung Kaiser Carls V. und des Heiligen Römischen Reichs}.

% \item \textit{Hamilton,Thomas}: Die Menschen und die Sitten in den Vereinigten Staaten von Nordamerika. Bd. 1-2. Mannheim 1834.

% \item \textit{Hansemann, David}: Preußen und Frankreich. Staatswirthschaftlich und politisch, unter vorzüglicher Berücksichtigung der Rheinprovinz. 2. verb. u. verm. Aufl. Leipzig 1834.

% \item \textit{Hantschke, Johann Carl Leberecht}: Hebräisches Uebungsbuch für Schulen. Leipzig 1823.

% \item \textit{Hegel, Georg Wilhelm Friedrich}: Grundlinien der Philosophie des Rechts, oder Naturrecht und Staatswissenschaft im Grundrisse. Hrsg. von Eduard Gans. In: Werke. Bd. 8. Berlin 1833.

% \item \textit{Heine, Heinrich}
% \begin{enumerate}
%     \item Reisebilder.
%     \item Über den Denunzianten. Eine Vorrede zum dritten Teile des ``Salons''.
%     \item Uber Ludwig Börne. Hamburg 1840.
% \end{enumerate}

% \item \textit{Hermes, Karl Heinrich}
% \begin{enumerate}
%     \item {[Leitartikel]} Köln, 23. Juni. In: Kölnische Zeitung, vom 24. Juni 1842.
%     \item {[Leitartikel]} Köln, 27. Juni. In: Kölnische Zeitung, vom 28. Juni 1842.
% \end{enumerate}

% \item \textit{Herodot von Halikarnaß}: Geschichte. Abth. 2. Bd. 8. Stuttgart 1831.

% \item \textit{Herwegh, Georg}: Brief an den König von Preußen. In: Leipziger Allgemeine Zeitung, vom 24. Dezember 1842.

% \item \textit{[Heß, Moses]}: [Vorbemerkung zu:] Die Berliner Familienhäuser. In: Rheinische Zeitung, vom
% 30. September 1842.

% \item \textit{Hey, Wilhelm}
% \begin{enumerate}
%     \item Erzählungen aus dem Leben Jesu für die Jugend, dichterisch bearbeitet. (Zu Olivier's Volksbilderbibel.) Hamburg 1838.
%     \item Fünfzig Fabeln für Kinder. Hamburg [1833].
%     \item Noch fünfzig Fabeln für Kinder. Hamburg [1837].
% \end{enumerate}

% \item \textit{Hinrichs, Hermann Friedrich Wilhelm}: [Rezension zu Bruno Bauer:] Die Posaune des jüngsten Gerichts über Hegel den Atheisten und Antichristen. Ein Ultimatum. Leipzig, 1841. In: Jahrbücher für wissenschaftliche Kritik. Nr. 52-55, März 1842.

% \item \textit{[Holbach, Paul-Henri-Dietrich d']}: Systême de la nature. Ou des loix du monde physique et du monde moral. Par Mirabaud. Part. 1-2. Londres 1770.

% \item \textit{Hugo, Gustav}: Lehrbuch eines civilistischen Cursus. Zweyter Band, welcher das Naturrecht, als eine Philosophie des positiven Rechts, besonders des Privat-Rechts, enthält. 4., sehr veränd. Ausg. Berlin 1819.

% \item \textit{Instruktion über die Verwaltung der Gemeinde- und Instituten-Waldungen in den Regierungs-Bezirken Coblenz und Trier in Folge des Gesetzes vom 24. Dezember 1816 und der Allerhöchsten Cabinets-Ordre vom 18. August 1835}. In: Amts-Blatt der Königl. Regierung zu Coblenz, vom 16. Oktober 1839.

% \item \textit{Jemand, Wilhelm}: Der ewige Jude. Didactische Tragödie. Iserlohn 1831.

% \item \textit{Jung, Alexander}
% \begin{enumerate}
%     \item Briefe über die neueste Literatur. Hamburg 1837.
%     \item Fragmente über den Ungenannten. In: Briefe über die neueste Literatur. Hamburg 1837.
%     \item Franz Ritter von Baader, In: Königsberger Literatur-Blatt, vom 1. Juni 1842.
%     \item Herbart. In: Königsberger Literatur-Blatt, vom 20. und 27. Oktober 1841.
%     \item Königsberg in Preußen und die Extreme des dortigen Pietismus. Braunsberg 1840.
%     \item Leo, Preußen und die Götheschen Wahlverwandtschaften. In: Königsberger Literatur-Blatt, vom 30. März 1842.
%     \item Stellung deutscher Journalistik. In: Königsberger Literatur-Blatt, vom 6. und 13. Oktober 1841.
%     \item Vorlesungen über die moderne Literatur der Deutschen. Danzig 1842.
%     \item Das Wesen des Christenthums von Ludwig Feuerbach. Leipzig. Otto Wiegand. 1841. In: Königsberger Literatur-Blatt, vom 24.November, 1., 8., 15. und 22. Dezember 1841.
%     \item Zur Orientirung über Schelling. In: Königsberger Literatur-Blatt, vom 17. November 1841.
% \end{enumerate}

% \item \textit{Juvenalis}: Satirae.

% \item \textit{Kaufmann, Peter}: Ueber die Nothwendigkeit und die Mittel, dem außerordentlichen Nothstande
% der Winzer am Nieder-Rheine, an der Mosel, Saar, Nahe und Ahr zu begegnen, und das ihnen bevorstehende Verderben abzuwenden. Vortrag geh. am 25. Sept. 1836 in d. 6. Generalvers. d. niederrheinischen landwirthschaftl. Vereins. In: Rhein- und Mosel-Zeitung, vom 9. und 11. November 1836.

% \item \textit{[Kay-Shuttleworth, James Phillipps]}: Recent measures, for the promotion of education in England. In: Eugène Buret: De la misère des classes laborieuses en Angleterre et en France; de la nature de la misère, de son existence, de ses effets, de ses causes, et de l'insuffisance des remèdes qu'on lui a opposés jusqu'ici; avec l'indication des moyens propres à en affranchir les sociétés. T. 1. Paris 1840.

% \item \textit{Knebel, Heinrich}: Französische Schulgrammatik für Gymnasien und Progymnasien. 2., verb. u. verm. Aufl. Koblenz 1836.

% \item \textit{Köster, Heinrich}: Kurze Darstellung der Dichtungsarten. In: Neunter Bericht über die höhere
% Stadtschule in Barmen. Barmen 1837.

% \item \textit{Kolb, Gustav}
% \begin{enumerate}
%     \item Die Communistenlehren. In: Allgemeine Zeitung, vom 11. Oktober 1842.
%     \item Herwegh. In: Allgemeine Zeitung, vom 3. Januar 1843.
%     \item {[Nachwort zu Richter:]} Die ständischen Berichte und die Rheinische Zeitung. In: Allgemeine Zeitung, vom 4. Januar 1843.
% \end{enumerate}

% \item \textit{Koran}.

% \item \textit{Kosegarten, Wilhelm}: Betrachtungen über die Veräusserlichkeit und Theilbarkeit des Landbesitzes mit besonderer Rücksicht auf einige Provinzen der Preußischen Monarchie. Bonn 1842.

% \item \textit{Krug, Friedrich Wilhelm}
% \begin{enumerate}
%     \item Kämpfe und Siege des jungen Wahlheim oder Lebensbilder aus dem Reiche des Wahren, Guten und Schönen. Bd. 1. Elberfeld 1833.
%     \item - Poetische Erstlinge und prosaische Reliquien. Barmen 1831.
% \end{enumerate}

% \item \textit{Krummacher, Friedrich Adolph}: Parabeln. Duisburg und Essen 1805.

% \item \textit{Kruse, Carl Adolf IV.}: Grundregeln der englischen Aussprache, nach Walker's System. Elberfeld 1837.

% \item \textit{Lamennais, Felicite-Robert de}: Paroles d'un croyant, 1833. Bruxelles 1834.

% \item \textit{Laube, Heinrich}: Geschichte der deutschen Literatur. Bd. 1-4. Stuttgart 1839-1840.

% \item \textit{Lessing, Gotthold Ephraim}: Eine Parabel. Nebst einer kleinen Bitte und einem eventualen Absagungsschreiben an den Herrn Pastor Goeze, in Hamburg.

% \item \textit{Lieth, Carl Ludwig Theodor}: Gedichte für das erste Jugend-Alter, zur Bildung des Herzen und Geistes. Aus Teutschlands besten Dichterwerken für Schule und Haus gesammelt. Th. 1-2. Crefeld 1834-1835.

% \item \textit{Lucian}: Werke. Übers, von August Pauly. Abth. 1. Bd. 2. Stuttgart 1827.

% \item \textit{Luther, Martin}
% \begin{enumerate}
%     \item Ermanunge zum fride auff die zwelff artickel der Bawrschafft ynn Schwaben. Auch widder die reubischen vnd mördisschen rotten der andern bawren. Wittemberg 1525.
%     \item Ein' feste Burg ist unser Gott.
% \end{enumerate}

% \item \textit{MacCulloch, John Ramsay}: Discours sur l'origine, les progres, les objets particuliers, et l'importance de l'economie politique. Contenant l'esquisse d'un cours sur les principes et la
% theorie de cette science. Geneve et Paris 1825. (Siehe auch Anm. 164.) 396

% \item \textit{Machiavelli, Niccolo}: Vom Staate oder Betrachtungen über die ersten zehn Bücher des Tit. Livius. Karlsruhe 1832.

% \item \textit{Meyen, Eduard}: [Rezension zu] Vorlesungen über die moderne Literatur der Deutschen von Alexander Jung. In: Rheinische Zeitung, vom 29., 30. und 31. Mai 1842.

% \item \textit{Moliere, Jean Baptiste}: Les Facheux.

% \item \textit{[Montesquieu, Charles de]}: De l'esprit des loix. Nouv. ed., revue, corr. et considerablement augm. T. 1-4. Amsterdam und Leipzig 1763.

% \item \textit{Mosen, Julius}: Ahasver. Episches Gedicht. Dresden und Leipzig 1838.

% \item \textit{Müntzer, Thomas}: Außlegung des andern vnterschyds Danielis deß propheten gepredigt auffm
% schlos zu Alstet vor den tetigen thewren Herzcogen vnd Vorstehern zu Sachssen durch Thomä Müntzer diener des wordt gottes. Alstedt 1524.

% \item \textit{Mündt, Theodor}: Madonna. Unterhaltungen mit einer Heiligen. Leipzig 1835.

% \item \textit{Owen, Robert}: The marriage system of the New Moral World; with a faint outline of the present
% very irrational system; as developed in a course of ten lectures. Leeds 1838.

% \item \textit{Platen-Hallermünde, August von}: Der romantische Oedipus. In: Gesammelte Werke. Stuttgart
% und Tübingen 1839.

% \item \textit{Plutarchus Chaeronensis}: Vitae parallelae. Solon.

% \item \textit{Pol, Johann}: Gedichte. Heedfeld 1837.

% \item \textit{Porter, George Richardson}: The progress of the nation, in its various social and economical relations, from the beginning of the nineteenth Century to the present time. Vol. 1-3. London 1836-1843.

% \item \textit{Proudhon, Pierre-Joseph}: Qu'est-ce que la propriete? Ou recherches sur le principe du droit et du gouvernement. Premier memoire. Paris 1840.

% \item \textit{Raumer, Friedrich von}: England im Jahre 1835. Th. 1-2. Leipzig 1836.

% \item \textit{[Richter]}: Die ständischen Berichte und die Rheinische Zeitung. In: Allgemeine Zeitung, vom 4. Januar 1843.

% \item \textit{Richter, Heinrich und Wilhelm}: Erklärte Haus-Bibel oder allgemein verständliche Auslegung der ganzen heiligen Schrift alten und neuen Testaments, nach vielen englischen, deutschen u.a. Auslegern bearbeitet. Bd. 1-6. Barmen und Schwelm 1834-1840.

% \item \textit{[Rousseau, Jean-Jacques]}: Du contrat social, ou principes du droit politique. Londres 1782.

% \item \textit{Rückert, Friedrich}
% \begin{enumerate}
%     \item Fünf Mährlein zum Einschläfern für mein Schwesterlein. Zum Christtag 1813. In: Gesammelte Gedichte. Erlangen 1834.
%     \item Die Verwandlungen des Ebu Seid von Seru'g öder die Maka'men des Hari'ri. Th. 1. [Stuttgart] 1826.
% \end{enumerate}

% \item \textit{[Rüge, Arnold]}: Der König von Preußen und die Socialreform. In: Vorwärts!, vom 27. Juli 1844.

% \item \textit{Savigny, Friedrich Carl von}: Der zehente Mai 1788. Beytrag zur Geschichte der Rechtswissenschaft.
% Berlin 1838.

% \item \textit{Schaper, Justus Wilhelm Eduard von}: [Reskripte] Koblenz, den 15. Dezember 1842. In: Rheinische Zeitung, vom 18. Dezember 1842.

% \item \textit{Schifflin, Philipp}: Anleitung zur Erlernung der französischen Spräche. 1. Cursus. 3., verb. Aufl. Elberfeld 1839. 2. Cursus. Elberfeld 1833. 3. Cursus. Elberfeld 1840.

% \item \textit{Schiller, Friedrich von}
% \begin{enumerate}
%     \item Die Götter Griechenlands.
%     \item Über naive und sentimentalische Dichtung.
%     \item Die Worte des Glaubens.
% \end{enumerate}

% \item \textit{Shakespeare, William}
% \begin{enumerate}
%     \item Hamlet.
%     \item Der Kaufmann von Venedig.
%     \item König Heinrich IV.
%     \item König Lear.
%     \item Ein Sommernachtstraum.
% \end{enumerate}

% \item \textit{[Sieyès, Emmanuel-Joseph]}: Qu'est-ce que le tiers-etat? 2. 6d., corr. [Paris] 1789.

% \item \textit{Sitzungs-Protokolle des sechsten Rheinischen Provinzial-Landtags}. Coblenz 1841.

% \item \textit{Smith, Adam}: An inquiry into the nature and causes of the wealth of nations. Vol. 1-2. London
% 1776. Vol. 3. Dublin 1776.

% \item \textit{Spinoza, Benedictus de}: Ethica.

% \item \textit{Stein, Lorenz von}: Der Socialismus und Communismus des heutigen Frankreichs. Ein Beitrag zur Zeitgeschichte. Leipzig 1842.

% \item \textit{Sterne, Laurence}: The Life and opinions of Tristram Shandy, Gentleman.

% \item \textit{Stier, Rudolf}: Christliche Gedichte. Basel 1825.

% \item \textit{Strauß, David Friedrich}
% \begin{enumerate}
%     \item Die christliche Glaubenslehre in ihrer geschichtlichen Entwicklung und im Kampfe mit der modernen Wissenschaft. Bd. 1-2. Tübingen und Stuttgart 1840-1841.
%     \item Das Leben Jesu. Bd. 1-2. Tübingen 1835-1836.
% \end{enumerate}

% \item \textit{Sue, Eugene}: Les mysteres de Paris.

% \item \textit{Tacitus, Publius Cornelius}: Historiae.

% \item \textit{Terentius Afer, Publius}: Andria.

% \item \textit{Tertullianus, Quintus Septimus Florens}: De carne Christi.

% \item \textit{Thucydides}: Geschichte des Peloponnesischen Kriegs. Abth. 1. Bd. 2. Stuttgart 1827.

% \item \textit{Tocqueville, Alexis de}: De la démocratie en Amérique. T. 1--2. Paris 1836.

% \item \textit{Uhland, Ludwig}: Die Rache.

% \item \textit{Ure, Andrew}: The philosophy of manufactures: or, an exposition of the scientific, moral, and commercial economy of the factory system of Great Britain. London 1835.

% \item \textit{Vanini, Julius Caesar}: Amphitheatrum aeternae providentiae divinomagicum. Lugduni 1615.

% \item \textit{Vedas}.

% #128
% \item \textit{Verordnung, wie die Zensur der Druckschriften nach dem Beschluß des deutschen Bundes vom 20sten September d.J. auf fünf Jahre einzurichten ist. Vom 18ten Oktober 1819}. In: Gesetz-Sammlung für die Königlichen Preußischen Staaten. Berlin 1819.

\subsubsection*{Last 10}

\item[129.] Virgil: \textit{Aeneis}.

\item[130.] François-Marie Arouet de Voltaire
\begin{enumerate}
    \item \textit{La Bible enfin expliquée}.
    \item \textit{L'enfant prodigue}.
\end{enumerate}

\item[131.] \textit{Vorstellungen der Direction des Vereins an verschiedene Behörden}. In: Mittheilungen des Vereins
zur Förderung der Weincultur an Mosel und Saar zu Trier. H. 4. Trier 1841.

\item[132.] John Wade
\begin{enumerate}
    \item \textit{British history}. London 1838.
    \item \textit{History of the middle and working classes; with a popular exposition of the economical and political principles}. London 1835.
\end{enumerate}

\item[133.] Wilhelm Weitling
\begin{enumerate}
    \item \textit{Das Evangelium eines armen Sünders}. Bern 1845.
    \item \textit{Garantien der Harmonie und Freiheit}. Vivis 1842.
\end{enumerate}

\item[134.] Christoph Martin Wieland: \textit{Der neue Amadis}.

\item[135.] J. Ch. F. Winkler: \textit{Harfenklänge, bestehend in einer metrischen Uebersetzung und Erläuterung von Ein und fünfzig ausgewählten Psalmen, und in einer Auswahl von evangelischen Gedichten und Liedern, nebst einem Anhang, in welchem nachträglich noch einige Psalmen geliefert werden}. Bärmen 1838.

\item[136.] Friedrich Ludwig Wülfing
\begin{enumerate}
    \item \textit{Ein Heftchen wackerer Gesänge}. 1832.
    \item \textit{Leier und Schwert oder Bienen, mit und ohne Stachel}. Barmen 1830.
    \item \textit{Jugendblüthem}. Barmen 1830.
\end{enumerate}

\item[137.] Heinrich Zoepfl: \textit{Grundsaetze des allgemeinen und des constitutionell-monarchischen Staatsrechts, mit Rücksicht auf das gemeingültige Recht in Deutschland, nebst einem kurzen Abrisse des deutschen Bundesrechtes und den Grundgesetzen des deutschen Bundes als Anhang}. Heidelberg 1841.

\item[138.] [Vincenz Jakob von Zuccalmaglio]: \textit{Montanus: Die Vorzeit der Länder Cleve-Mark, Jülich-Berg und Westphalen}. Bd. 1. 2. Aufl. Solingen und Gummersbach 1837. Bd. 2. Solingen 1839.

\end{enumerate}
