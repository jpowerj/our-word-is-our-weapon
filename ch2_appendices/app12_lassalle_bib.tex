Bert Andréas' 1981 bibliography \parencite{andreas_ferdinand_1981} provides an exhaustive listing of Lassalle's literary output available in any language up to the time of publication, which enabled us to compile digitized versions of the subset of all known German versions (original or in translation) of Lassalle's writings, used for the analysis in Section \ref{sec:marxvlassalle}. As a sample of the information in this bibliography, the ``base'' entry (i.e., excluding the additional entries representing variants and translations of the original publication) for the first ten and last ten records are reproduced below:

\paragraph{First 10}

\begin{itemize}
    \item \textbf{A1.1}: \textit{REISEBESCHREIBUNG VON MEINEM LIEBLINGSWINKEL BIS ZUR STUBENTOR}
    \begin{itemize}
        \item N VI S. 6--10
        \item Schulaufsatz vom August 1840
    \end{itemize}
    
    \item \textbf{A2.1}: \textit{[FERDINAND LASSALLES TAGEBUCH]}
    \begin{itemize}
        \item Herausgegeben und mit einer Einleitung versehen van Paul Lindau. Breslau, Schlesische Buchdruckerei, Kunst- und Verlags-Anstalt vormals S. Schott-laender, 1891, in-16, 259 S.
        \item Das Tagebuch wurde geführt von Januar 1840 bis zum Sommer 1841. Lindau hat den Text zuerst veröffentlicht in \textit{Nord und Süd}, Bd. LVII, Heft 169--170. Zwei von ihm ausgelassene längere Eintragungen (von etwa Pfingsten und Sommer 1841) hat Gustav Mayer nach dem Manuskript veröffentlicht in N I, S. 54--63. Mayers Handexemplar obiger Ausgabe mit den Verbesserungen des von Lindau teilweise korrumpierten Textes ist jetzt im IISG.
        \item Vgl. \register{L.B43} und \register{L.C170}.
        \item Archives: BA, FUB, IISG, IMLB, SUBD, SUBH
    \end{itemize}
    
    % \item[A2.2] \textit{[FERDINAND LASSALLE. TAGEBUCH DES LEIPZIGER HANDELSSCHÜLERS. MAI 1840 BIS MAI 1841]}
    % \begin{itemize}
    %     \item Berlin-Wilmersdorf, Verlag der Wochenschrift ``Die Aktion'', (Franz Pfemfert), 1918, in-8, 88-[6] S.
    %     \item Archives: BA FUB IISG IMLB KMH UBK
    % \end{itemize}
    
    % \item[A2.3] \textit{[FERDINAND LASSALLES TAGEBUCH]}
    % \begin{itemize}
    %     \item Herausgegeben und mit einem tachjort versehen von Friedrich Hertneck. Berlin, Weltgeist-Bücher Verlags-Gesellschaft, [1926], in-16, 140-[l] S.
    %     \item Archives: ASD BA HBSA IMLB SUBD
    % \end{itemize}
    
    % \item[A2.4] \textit{Weltgeist-Bücher} Nr. 152-153. Der Lindausche Text verbessert nach Gustav Mayers Handexemplar.
    
    % \item[A2.T1] (\textit{Russisch}): St. Petersburg, Zvonarev, 1901, 263 S. (herausgegeben von B.N. Jevonarev); Petrograd 1918, 168 S.; Petrograd, Izd. Petrogr. Soviet Rab. i Krasnoarm., 1919, 167-[1] S.
    		
    \item \textbf{A3.1}: \textit{ÜBER DIE ERKLÄRUNG DER HERREN KOLLOFF, SCHUSTER UND HAMBERG.}
    \begin{itemize}
        \item \textit{Breslauer Zeitung}, 25. September 1841, Nr. 224, S. 1606.
        \item N VI S. 31--33
        \item Parteinahme für Heinrich Heine gegen Salomon Strauß, datiert vom 24. September 1841. Zugeschrieben von Gustav Mayer.
        \item Vgl. \register{L.C552}.
        \item Archives: UBW
    \end{itemize}
    
    \item \textbf{A4.1}: \textit{ZUR ERKLÄRUNG DES HERRN DR. DAVIDSON IN NO. 222 DER LEIPZIGER ALLGEMEINEN ZEITUNG.}
    \begin{itemize}
        \item \textit{Bulletin des Leo Baeck Instituts}, Tel Aviv, 1966, Jahrg. IX, Nr. 36, S. 335--338.
        \item über Rabbinatsstreitigkeiten in der Breslauer jüdischen Gemeinde. Der Artikel war für eine unbekannte Zeitung bestimmt, die ihn nicht aufnahm. Gustav Mayer hat das Manuskript koimentiert in \register{L.C557}. Der Artikel wurde veröffentlicht zusammen mit \register{L.A9a} und mit einem Kommentar von Alex Bein in \register{L.C106}.
        \item Archives: BA BM JNUL LR OCH
    \end{itemize}
    
    \item \textbf{A5.1}: \textit{WIE KONNTEN DIE ALTEN BEI IHREM AUSGEBILDETEN RECHTSGEFÜHL DIE SKLAVEREI DULDEN?}
    \begin{itemize}
        \item N VI S. 10--12
        \item Schulaufsatz, den Gustav Mayer auf ``1842/43'' datiert.
    \end{itemize}
    
    \item \textbf{A6.1}: \textit{KANN DIE REALBILDUNG DIE KLASSISCHE BILDUNG ERSETZEN?}
    \begin{itemize}
        \item N VI S. 12--16
        \item Schulaufsatz, den Gustav Mayer auf ``1842/43'' datiert.
    \end{itemize}
    
    \item \textbf{A7.1}: \textit{STOIKER ODER EPIKUREER}
    \begin{itemize}
        \item N VI S. 17--20
        \item Schulaufsatz, den Gustav Mayer auf ``1842/43'' datiert.
    \end{itemize}
    
    \item \textbf{A8.1}: \textit{ANSPRACHE AN LESSINGS GEBURTSTAG}
    \begin{itemize}
        \item N VI S. 20--23
        \item Schulaufsatz, den Gustav Mayer auf ``Januar 1842 (oder auch 1843)'' datiert.
    \end{itemize}
    
    \item \textbf{A9.1}: \textit{DER VIELWISSER}
    \begin{itemize}
        \item N VI S. 23--27
        \item Schulaufsatz, den Gustav Mayer auf ``1842/43'' datiert.
    \end{itemize}
    
    \item \textbf{A9a.1}: \textit{[SPOTTGEDICHT AUF DEN JÜDISCHEN LEHR- UND LESEVEREIN IN BRESLAU]}
    \begin{itemize}
        \item \textit{Bulletin des Leo Baeck Instituts}, Tel Aviv 1966, Jahrg. IX, Nr. 36, S. 338--341. Gustav Mayer zitiert einige Zeilen in \register{L.C557} und datiert das Gedicht auf ``1843''. Das Gedicht wurde zusammen mit \register{L.A4} vollständig und mit einem Kommentar veröffentlicht von Alex Bein in \register{L.C106}.
        \item Archives: BA BM JNUL LBN LR OCH
    \end{itemize}
    
    \item \textbf{A10.1}: \textit{GRUNDZÜGE ZU EINER CHARAKTERISTIK DER GEGENWART MIT BESONDERER BERÜCKSICHTIGUNG DER HEGEL'SCHEN PHILOSOPHIE}
    \begin{itemize}
        \item \textit{Zeitschrift für moderne Philosophie}, Breslau 1843, Nr. 1.
        \item N VI S. 55--74.
        \item Der im Sommer 1843 geschriebene Aufsatz ist gezeichnet ``F. Lassal''. Er füllt in der handschriftlich hergestellten Studentenzeitschrift die ganze erste Nummer (44 1/4 Spalten), ohne damit beendet zu sein; weitere Nummern sind erschienen, jedoch nicht erhalten geblieben. Lassalle veröffentlichte in ihnen noch mindestens einen Artikel \textit{Zur Religionsphilosophie des Christenthums}.
        \item Das von Gustav Mayer für seinen Abdruck des Textes benutzte Exemplar der ersten Nummer der Zeitschrift befindet sich im Lassalle-Nachlaß. Die Wroctawer Bibliotheken teilten mit, keine Nummern der Zeitschrift zu besitzen. Die Nachlässe der Mitherausgeber der Zeitschrift, R. von Gottschall und M. von Wittenburg, gelten als verschollen.
    \end{itemize}
    
    \item \textbf{A10a.1}: \textit{[ÜBER DEN HANDEL UND OBER DEN WEBERAUFSTAND]}
    \begin{itemize}
        \item N I S. 99--105
        \item Brief an den Vater, vom 12. Juni 1844.
        \item Nach seiner Übersiedelung an die Berliner Universität schickte Lassalle mehrere ausführliche Manuskripte zu einzelnen Themen in Briefform an seine Eltern und Freunde. In diesen ``Manuskriptbriefen'' fand die Selbstverständigung des Studenten ihren Niederschlag. Gustav Mayer veröffentlichte vier dieser Texte, die bewahrt geblieben waren (\register{L.A10a}, \register{L.A10b}, \register{L.A10c} und \register{L.A12a}). Ebenfalls als Manuskriptbriefe anzusehen sind die Texte \register{L.A24a} und \register{L.A25b} sowie die an Sofija Sontzova gerichtete sogenannte ``Seelenbeichte'' vom Oktober 1860 (Brief Nr. 7 in \register{L.B11}).
    \end{itemize}
    
    \item \textbf{A10b.1}: \textit{[ÜBER JUDENTUM UND GESCHICHTE]}
    \begin{itemize}
        \item N I S. 106--114
        \item Brief an die Mutter, vom 30. Juli 1844
    \end{itemize}
    
    \item \textbf{A10c.1}: \textit{[ÜBER ADEL, STAAT, INDUSTRIE UND KOMMUNISMUS]}
    \begin{itemize}
        \item N I S. 114--136
        \item Brief an den Vater, vom 6. September 1844. Von Lassalle ``Brief über die Industrie'' genannt (in \register{L.A12}).
    \end{itemize}
\end{itemize}

\paragraph{Last 10}

\begin{itemize}
    \item[\textbf{A91.1}] \textit{DER HOCHVERRATES-PROZESS WIDER FERDINAND LASSALLE VOR DEM STAATS-GERICHTS-HOFE ZU BERLIN, AM 12. MÄRZ 1864. NACH DEM STENOGRAPHISCHEN BERICHT.}
    \begin{itemize}
        \item Berlin, Verlag von Reinhold Schlingmann, [Druck von R. Gensch in Berlin, Kronenstraße 36], 1864, in-8, 78 S.
        \item RS II S. 754--830
        \item GRS IV S. 61--174
        \item In der Anklageakte (S. 4--10) wegen der Veröffentlichung von \register{L.A77} stellt das Gericht fest, die Auflage habe 16000 Exemplare betragen, von denen insgesamt 3026 beim Verleger, dem Verfasser und einem Expedienten beschlagnahmt wurden. Die Verteidigungsrede Lassalles auf S. 36--73. Eine Eingabe Lassalles an den Anklagesenat, vom 29. November 1863, druckt G. Mayer in N VI S. 384--391. Die Broschüre erschien in einer Auflage von 1200 Exemplaren (vgl. Lassalle an Schlingmann, 25. Mai 1864). Ein langer Brief der Gräfin Hatzfeldt an Mathilde Anneke vom 23. März 1864 (in WHi, Anneke Papers) schildert ausführlich Lassalles Auftreten vor Gericht und seine Rede.
        \item Archives: ASD BA IMLB SUBD UBBa UBK
    \end{itemize}
    
    \item[\textbf{A92.1}] \textit{PROCLAMATION}
    \begin{itemize}
        \item RS II S. 905--907
        \item GRS IV S. 268--271 		
        \item Datiert und unterzeichnet ``Berlin, den 17. März 1864. Der Präsident des Allg. Deutsch. Arbeiter-Vereins. Ferdinand Lassalle.''
        \item Lassalle gibt bekannt, daß künftig in Abweichung von den Bestimnungen des Textes \register{L.A68} die rechtsrheinischen Gemeinden auf Grund ihrer ``starken Mitgliederzahl'' selbst ihre Bevollmächtigten Vorschlägen. Gleichzeitig wird Carl Klings zu Lassalles ``Kommissar'' ernannt, der die Durchführung dieser Anordnung zu organisieren hat.
    \end{itemize}
    
    \item[\textbf{92a.1}] \textit{[VERFÜGUNG]}
    \begin{itemize}
        \item In-8, [1] S. Handschrift vervielfältigt.
        \item Die aus Berlin vom 7. Mai 1864 datierte Verfügung ist an den Vereins-Kassierer Gustav Lewy gerichtet und ordnet die monatliche Auszahlung des Sekretär-Gehalts an Eduard Willms an. Es wurde kein Exemplar gefunden, aber das Nachlaß-Repertorium sagt dazu: ``Eigenh. Ausfertigung Lassalles und Faksimile-Umdruck in zahlreichen Exemplaren, je 1 S. 8°'' (V, 11 in \register{L.C104}).
        \item Demnach ist möglicherweise die Versendung unterblieben.
    \end{itemize}
    
    \item[\textbf{A93.1}] \textit{[REDE IN DER LEIPZIGER GEMEINDE DES ADAV]}
    \begin{itemize}
        \item N VI S. 274--282
        \item Fragment einer Nachschrift (mit Verbesserungen von Lassalles Hand) der Rede, welche Lassalle am 9. Mai 1864 in Leipzig gehalten hat. G. Mayer ergänzt das unvollständige Manuskript aus den Versammlungsberichten des Hamburger \textit{Nordstern} (2. Mai 1864, Nr. 258, S. 2) und des Leipziger \textit{Adler. Zeitung für Deutschland} (11. Mai 1864). Große Teile dieser Rede sind identisch mit der Rede \register{L.A94}.
    \end{itemize}
    
    \item[\textbf{A93a.1}] \textit{HERRN OTTO DANNER IN LEIPZIG.}
    \begin{itemize}
        \item  In-8, [1] S. Handschrift vervielfältigt.
        \item Von Danmer geschrieben und datiert ``Leipzig, 11. Mai 1864'' und von Lassalle unterzeichnet. Lassalle ernennt Danner ``für die Dauer meiner Abwesenheit von Berlin zum Vice-Präsidenten'' des ADAV und überträgt ihm ``alle mir selbst zustehenden Funktionen und Befugnisse''. In einem kurzen Vorsatz ersucht Danner ``Geehrte Redaction'' um Aufnahme des Textes. Er übersandte das Zirkular mit einem Begleitschreiben vom 12. Mai 1864 (Exemplar in HA) an die Bevollmächtigten und Vorstandsmitglieder.
        \item Archives: HA, IISG
    \end{itemize}
    
    \item[\textbf{A94.1}] \textit{DIE AGITATION DES ALLG. DEUTSCHEN ARBEITERVEREINS UND DAS VERSPRECHEN DES KÖNIGS VON PREUSSEN. EINE REDE GEHALTEN AM STIFTUNGSFEST DES ALLGEMEINEN DEUTSCHEN ARBEITER-VEREINS ZU RONSDORF AM 22. MAI 1864 VON FERDINAND LASSALLE}.
    \begin{itemize}
        \item Berlin, Verlag von Reinhold Schlingmann, [Druck von R. Gensch in Berlin, Kronenstr.36,] 1864, in-8, 52 S.
        \item RS II S. 841--872
        \item GRS IV S. 187--229
        \item Die Broschüre erschien in einer Auflage von 2000 Exemplaren (vgl. Schlingmann an Lassalle, 28. Mai 1864) gegen Ende Juni 1864 (vgl. Willms an Lassalle, 19. Juni 1864). Die Korrektur hat Bucher gelesen (vgl. Bucher an Lassalle, 6. Juni 1864). Das unvollständige Konzept zu dieser Rede hat G. Mayer gedruckt in N VI S. 282--284. Lassalle hat dieselbe Rede vorher in verschiedenen anderen rheinischen Städten gehalten: in Düsseldorf am 13. Mai, in Solingen und Barmen am 14. Mai, in Köln am 15. Mai, in Duisburg am 16. Mai und in Wermelskirchen am 18. Mai. Im Anhang der Broschüre sind mehrere Versanmlungsberichte nachgedruckt (S. 42--52), darunter zwei von Lassalle selbst verfaßte (vgl. \register{L.A95}, \register{L.A96}). Der Anhang ist auch in allen Nachdrucken enthalten. Das Exemplar in IMLB befindet sich in einem Konvolutband aus dem Besitz Clara Zetkins.
        \item Vgl. \register{L.A93}.
        \item Archives: BA, IISG, IMLB, SUBH, UBK, ZBZ
    \end{itemize}
    
    \item[\textbf{A95.1}] \textit{WERMELSKIRCHEN, DEN 19. MAI}
    \begin{itemize}
        \item \textit{Nordstern}, Hamburg, 28. Mai 1864, Jahrg. V, Nr. 259, S. 3.
        \item RS II S. 875--878
        \item GRS IV S. 232--237
        \item Anonymer Bericht über die Agitationsversanmlung in Wermelskirchen am 18. Mai 1864, vor der Lassalle die Rede \register{L.A94} gehalten hat. Der Bericht enthält ein ``Lied zur Abholung des Präsidenten F. Lassalle'', dessen ``tiefe Innigkeit'' und ``Charakter des echten alten Volksgesangs'' hervorgehoben werden. Die Verfasserschaft des Berichts ist belegt durch den eigenhändigen Entwurf Lassalles und einen von ihm bearbeiteten Korrekturabzug im Nachlaß, denen die Manuskripte der beiden Gedichte beigelegt sind (V, 12 in \register{L.C104}).	
        \item Der Bericht ist, soweit nicht anders bemerkt, in allen Ausgaben von \register{L.A94} abgedruckt.
        \item Archives: IISG
    \end{itemize}
    
    \item[\textbf{A96.1}] \textit{RONSDORF, 23. MAI}
    \begin{itemize} 
    	\item \textit{Der Adler}, Leipzig, 25. Mai 1864
    	\item RS II S. 878--881
    	\item GRS IV S. 237--240
    	\item Anonymer Bericht über die Agitationsversanmlung in Ronsdorf am 22. Mai 1864, vor der Lassalle die Rede \register{L.A94} gehalten hat. Die Verfasserschaft wird Lassalle zugeschrieben von Vahlteich (\register{L.C893}, S. 8) und von Bernstein (RS II, S. 839). Der Adler konnte nicht aufgefunden werden. Der Bericht ist, soweit nicht anders bemerkt, in allen Ausgaben von \register{L.A94} abgedruckt.
    \end{itemize}
    
    \item[\textbf{A97.1}] \textit{ERWIDERUNG}
    \begin{itemize}
        \item \textit{Neue Preußische (Kreuz-) Zeitung}, Berlin, 19. Juni 1864, Nr. 141, 1. Beilage S. 1.
        \item RS III S. 270--282
        \item GRS V S. 365--381
        \item Datiert und unterzeichnet ``Bad Ems, den 2. Juni 1864. F. Lassalle.'' Die Erwiderung antwortet auf die lange Besprechung von \register{L.A87} in der \textit{Kreuz-Zeitung} (vgl. \register{L.C907}). Die Aufnahme der Entgegnung in das konservative Organ erfolgte, nach anfänglicher Weigerung (Brief der Redaktion an Lassalle, 8. Juni 1864), erst auf Intervention des ehemaligen Chefredakteurs der Zeitung, Hermann Wagener. Der Bismarck sehr nahestehende Wagener war der anonyme Verfasser der Besprechung.
        \item Auf Lassalles Instruktion kaufte der Vereinssekretär Willms 50 Exemplare der Beilage (oder bestellte 50 Sonderabzüge des Textes), um sie an die Bevollmächtigten zu versenden. Vgl. Lassalle an Willms, 15. Juni 1864, und Willms an Lassalle, 19. Juni 1864.
        \item Vgl. hierzu auch \textit{Der Artikel des Herrn Lassalle in der Kreuzzeitung vom 19. Juni mit einigen Randbemerkungen} in \textit{Deutsche Arbeiterzeitung}, Leipzig, 15. Juli 1864, Nr. 16, S. 123--127.
        \item Archives: ASD, IZD
    \end{itemize}
    
    \item[\textbf{A98.1}] \textit{PROZESS LASSALLE}
    \begin{itemize}
        \item \textit{Düsseldorfer Zeitung}, 29. Juni 1864, Nr. 176, S. 2--3; 30. Juni, Nr. 177, S. 2--3; 1. Juli, Nr. 178, S. 2--3.
        \item RS II S. 677--706
        \item GRS III S. 405--444 
        \item Bericht über den in Düsseldorf am 27. Juni 1864 in zweiter Instanz verhandelten Prozeß wegen der Broschüre \register{L.A73}. Der Bericht erwähnt eingangs, daß am 21. Oktober 1863 von der Gesamtauflage von 10000 Exemplaren beim Verleger 1034 und noch ``einige 20 Exemplare'' bei Buchhändlern in Berlin und Düsseldorf beschlagnahmt wurden. Der obige Titel erscheint nur in Nr. 177 und 178 der \textit{Düsseldorfer Zeitung}, die Lassalles Verteidigungsrede enthalten. Den Prozeßbericht schrieb der Lassalle befreundete Redakteur Paul Lindau auf Grund seiner stenographischen Nachschrift der Rede, die er Lassalle vorlas, der ihm nach seinem Konzept Korrekturen und Zusätze diktierte. Lindau hat das Konzept in \register{L.C502} (S. 13--22) veröffentlicht. Das Original befindet sich jetzt in SLD. Das Urteil und eine Zusammenfassung der ausführlichen Urteilsbegründung in der \textit{Düsseldorfer Zeitung} vom 3. Juli 1864, Nr. 180, S. 3.
        \item Archives: SUBD
    \end{itemize}
    
    \item[\textbf{A99.1}] \textit{CIRCULAR AN SÄMTLICHE VORSTANDS-MITGLIEDER.}
    \begin{itemize}
        \item Druck von J. Draeger's Buchdruckerei (C. Feicht) in Berlin, o.J. in-8, 15-1 S. 
        \item RS II S. 911--927
        \item GRS IV S. 276--298
        \item Datiert und unterzeichnet ``Rigi-Kaltbad, 27. Juli 1864. Der Präsident des Allg. deutschen Arbeitervereins. Ferdinand Lassalle.'' Lassalle stellt ausführlich seine verschiedenen Konflikte mit Vahlteich dar (vgl. \register{L.A86}) und begründet seine unausgesprochene Forderung, Vahlteich aus dem ADAV auszuschließen. Anlaß zu dem ungewöhnlich umfangreichen Zirkular und dem scharfen Ton gegen Vahlteich war dessen Antrag vom 28. Juni 1864 (RS II S. 909--910; GRS IV S. 274--275) zur Generalversammlung des ADAV für 1864, der auf eine Einschränkung der Präsidialbefugnisse zugunsten derer des Vorstandes und auf eine demokratische Beschickung der Generalversammlung abzielte. Lassalle fügte dem Manuskript eine Privatinstruktion an den Vereinssekretär Willms bei (RS II S. 928--930; GRS IV S. 298--301), aus der hervorgeht, daß er ihm zugleich zwei weitere Zirkulare an die Vorstandsmitglieder zur Vervielfältigung und Versendung schickte. Sie betrafen die Ernennung von B. Becker und von Schweitzer zu Vorstandsmitgliedern und die Ernennung von Kassenrevisoren. Keines dieser beiden Zirkulare wurde aufgefunden.
        \item Lassalles letztes Zirkular erschien am 3. August 1864 im Druck (Vgl. Willms an Lassalle, 3. August 1864) in einer Auflage von wahrscheinlich 100 Exemplaren (vgl. Lassalle an Willms, 27. Juli 1864). Vahlteich antwortete darauf mit einem Rundschreiben an die Vorstandsmitglieder vom 11. August 1864 (RS II S. 930; GRS IV S. 302).
        \item Vgl. \register{L.B10}, S. 245--256.
        \item Archives: BIF, SBB*
    \end{itemize}
    
    \item[\textbf{A100.1}] \textit{DIES IST MEIN TESTAMENT.}
    \begin{itemize}
        \item Großenhain 1889 in \register{L.C441} (S. 12--15)
        \item RS II S. 956--958
        \item GRS IV S. 337--339
        \item Datiert und unterzeichnet ``Eigenhändig geschrieben und unterschrieben: Genf 27 August 1864 Ferdinand Lassalle''. Vermacht u.a. ``Die gelehrten und schriftstellerischen Aufsätze und Notizen'' sowie ``Das Eigenthum an meinen sämmtlichen schriftstellerischen und gelehrten Werken [...] Herrn Lothar Bucher.'' In der für den ADAV entscheidenden Verfügung empfiehlt Lassalle die Wahl des Frankfurter Bevollmächtigten Bernhard Becker zu seinem Nachfolger, mit dem Rat: ``Er soll an der Organisation festhalten! Sie wird den Arbeiterstand zum Sieg führen!'' Der Abdruck in \register{L.C441} erfolgte nach der Abschrift in Holthoffs Nachlaß, die der Gräfin Hatzfeldt von den Genfer Behörden zur Verfügung gestellt worden war. Der Abdruck wurde mit dem Original im Genfer Staatsarchiv (Jur. Civ., AAQ \textit{Testaments}, vol. 13, No. 116) verglichen. Er weicht von ihm nur in unbedeutenden Einzelheiten ab, wie Auflösung von Abkürzungen, Umwandlung ausgeschriebener Zahlen in arabische Ziffern, Einfügung fehlender Interpunktion u.ä. Kohut druckte denselben Text noch einmal im selben Jahre in \register{L.C442} (S. 190--192).
        \item Damit ist Na'amans Hypothese hinfällig, ein Kopistenirrtum habe obigen Text an die Stelle der von Na'aman vermuteten Formulierung ``Er soll die Organisation festhalten'' gesetzt (vgl. \register{L.C650} und \register{L.C652}).
    \end{itemize}
\end{itemize}

The source abbreviations for this subset are as follows:
\begin{itemize}
    \item[ASD] Archiv für soziale Demokratie, Bonn, Germany
    \item[BA] Sammlung Bert Andréas, im Institut Universitaire de Hautes Etudes, Genf [Geneva], Switzerland
    \item[BIF] Biblioteca dell'Istituto Giangiacomo Feltrinelli, Milan, Italy
    \item[BM] British Library, im British Museum, London, UK
    \item[FUB] Bibliothek der Freien Universität, Berlin, Germany
    \item[HA] Herwegh Archiv, im Dichter Museum, Liestal, Switzerland
    \item[HBSA] Hamburger Bibliothek für Sozialgeschichte und Arbeiterbewegung, Hamburg, Germany
    \item[IISG] Internationaal Instituut voor Sociale Geschiedenis, Amsterdam, The Netherlands
    \item[IMLB] Institut für Marxismus-Leninismus, Berlin, Germany
    \item[IZD] Institut für Zeitungsforschung, Dortmund, Germany
    \item[JNUL] Jewish National and University Library, Jerusalem, Occupied Palestinian Territories
    \item[KMH] Karl Marx Haus, Bibliothek, Trier, Germany
    \item[LBN] Leo Baeck Institute, New York, NY, USA
    \item[LR] Ludwig Rosenberger (Privatsammlung), Chicago, IL, USA
    \item[MS] Manuskript
    \item[N] \register{L.B65}
    \item[OCH] Hebrew Union College Library, Cincinnati, OH, USA
    \item[SBB*] Stadtbibliothek, Sondersammlungen, Berlin, Germany
    \item[SUBD] Stadt- und Landesbibliothek, Düsseldorf, Germany
    \item[SUBH] Staats- und Universitätsbibliothek, Hamburg, Germany
    \item[UBBa] Universitäts-Bibliothek, Basel, Switzerland
    \item[UBK] Universitäts-Bibliothek, Köln [Cologne], Germany
    \item[UBW] Uniwersytet Wroc\l{}awski Bibliotek, Wroc\l{}aw [Breslau], Poland
    \item[ZBZ] Zentralbibliothek, Zürich, Switzerland
\end{itemize}
