We use newly-developed tools emerging out of the ``neural revolution'' in computational linguistics to assess the interplay between major political events and the texts which both influence and are influenced by them. We illustrate throughout how these tools enable new insights---for example, by providing evidence for or against differing interpretations of an event derived from statistically-principled and representative analyses of textual evidence---but do not thereby diminish the importance of deep ``manual'' engagement with the evidence and creativity on the part of the researcher.

We start with an analysis of the 1789 ``grievance books'' which, requested by King Louis XVI as a way to assess the roots of discontent across the various estates and geographic regions of France, provide an unprecedented source of data for historians: a comprehensive survey of public sentiment on the immediate eve of a mass revolutionary upheaval. Since there are 60k \textit{cahiers}, however, far too many for any one researcher or team of researchers to read and integrate, we show how a computational-linguistic approach can minimize selection bias and thus enable an understanding of the origins of the French Revolution based on a representative analysis of the full body of available evidence.

From this cross-sectional study of 1789 we move to a pair of time-series analyses of the development and diffusion of socialist ideology, from the early days of the Enlightenment to the fall of the Soviet Union. The first study examines 19th-century socialist thought from its Enlightenment and Romantic origins to the ascendance of Marxism, using computational tools to (a) trace the relative influence of earlier German, French, and British thought on the formation of Marx's critique of capitalism, and then (b) analyze his polemical interventions into socialist discourse and how they succeeded in moving it away from the moral-philosophical frame of Robert Owen and Wilhelm Weitling and towards one employing a political-economic vocabulary rooted in Adam Smith and David Ricardo. The second study picks up where the first leaves off, tracing the global diffusion of ``scientific'' Marxism during the 20th century through the cultural-diplomatic initiatives of the Soviet Union. We show how the text-analytic methods of the previous study can be combined with more standard regression methods, in this case applied to data on global book publication and distribution, to complement the IR literature on foreign military intervention with a quantitative study of Soviet ``ideological intervention''.

%Our final study combines the cross-sectional and time-series approaches of previous chapters, examining how two major organizations during the First Palestinian Intifada---Hamas and the United National Command---competed with and responded to one another in their communiqués attempting to steer the course of the uprising. We show how text-reuse and paraphrase detection algorithms, in combination with the topic-detection methods of previous chapters, reveal a systematic adoption of Hamas' language and themes in the communiqués of the UNC as the latter's influence declined.

The dissertation is a collection of three studies, arranged in chronological order by subject and united by the common theme of how recent breakthroughs in computational linguistics---the ``Neural Revolution''---can be employed to gain a deeper understanding of historical events. The four chapters are titled as follows:

\begin{enumerate}
\item ``Simultaneous and Systematic Abolition''?: Text-Analyzing the 1789 \textit{Cahiers de Dol\'{e}ances}
\item The Geometry of Political Discourse: Mapping 19th Century Capitalist Critique Using Tools from Natural Language Processing
\item Quantifying Cultural Diplomacy: The Translation and Diffusion of Marxism from the Communist Manifesto to the Cold War
\item \textit{Fi Hadal Habs}: Strategic Speech Acts and Adaptation in the First Intifada's War of Words
\end{enumerate}

The motivating research question is: what can we learn, as social scientists, from data in the form of natural language text rather than the ``standard'' numeric data of empirical social-scientific inquiry? More importantly, perhaps, what can we \textit{not} learn using this approach? This work aims to address both of these questions---the possibilities and limitations of quantitative text-analysis in social science---by way of a chronological exploration of four historical epochs: the French Revolution, the 19th century genesis and diffusion of socialism, Cold War competition for influence in the Third World, and the eclipse of secular Soviet-aligned groups by political-Islamic forces in the First Palestinian Intifada. In each study we take care to emphasize how computational tools can \textit{augment}, but not replace, researchers' investigative capacities. Creativity, critical thinking, and the ability to discover novel connections among disparate data points all remain central to the endeavor.