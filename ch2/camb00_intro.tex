Perhaps the single key driver of the development of Marx's thought over the course of his life was his almost obsessive impulse to engage in polemics with those he viewed as obstacles to (in his early years) or rivals to (in later years) his assumption of the throne as the intellectual leader of the 19th century European socialist movement. We have already encountered this tendency in previous sections, when discussing the development of his thought with respect to the ``stationary targets'' of his criticism---namely, the already-established big names in political economy, early socialism, and philosophy such as Adam Smith, David Ricardo, Henri de Saint-Simon, and Hegel. In this section we turn to Marx's ``forward-facing'' polemics against his various interlocutors, his critique of the ``moving targets'' vying with him for intellectual influence over the quickly-growing European socialist movements of the era.

Central to this section is the move from \textit{influences} to \textit{interlocutors}. While Marx could isolate himself in his studies of the former for as long as he needed to develop cogent critiques, the European socialist movement moved at its own pace, requiring his critiques of the latter to be not only cogent but \textit{timely} as well. Engels, for example, urged Marx in 1845 to complete and publish his already-in-progress politial-economic tract as soon as possible, emphasizing that ``people's minds are ripe and we must strike while the iron is hot''\footnote{\mew{27}, p. 16.}. The urgency heightened even more when, in the ensuing year, the prominent French socialist Pierre-Joseph Proudhon published his own such tract titled \textit{Système des contradictions économiques}. As Keith Tribe puts it in his insightful analysis of Marx's subsequent polemics with Proudhon, ``the publication of Proudhon’s \textit{Système} galvanised [Marx] into writing a shorter work so that he might stake his claim to being Western Europe's foremost `radical political economist'. [Marx's 1847] \textit{Misère de la philosophie} is primarily a bid for market leadership, and it certainly reads that way.'' (\cite{tribe_economy_2015}, p. 224) 

Similarly, after having spent a decade in exile in London desperately working to maintain his relevance within the growing socialist movement of his home country, the 1859 publication of an attack on him in a German newspaper persuaded him to drop his political-economic studies for nearly two years in order to craft a response which would quell the ``foggy gossip of the [1848] refugees''\footnote{\mew{30}, p. 17.}. The attack, penned by a minor figure from the 1848 Frankfurt Assembly named Karl Vogt, cajoled Marx into working full-time to collect evidence against Vogt, culminating in a 208-page work, \textit{Herr Vogt}, for which he was never able to find a German publisher. Thus, in the remainder of the section, we take seriously the centrality of these ``wars of ideas'' in the development of Marx's thought by turning to an analysis of his illocutionary moves---what he was \textit{doing} with these polemical interventions, and how exactly he positioned himself within ideological spaces, in a way which eventually succeeded in crowding out all other contenders.

While some polemical episodes are not considered herein---for example, his 1847 polemic against Karl Heinzen after the latter's attack on Engels in the \textit{Deutsche-Brusseler Zeitung}\footnote{Culminating in the article ``Moralising Criticism and Critical Morality. A Contribution to German Cultural History. Contra Karl Heinzen'', published in the same newspaper in October and November of that year. \mecw{6}, p. 312.}, or his 1865 debate with his fellow International Working Men's Association (IWMA) General Council member John Weston\footnote{Culminating in an 1865 address to the IWMA in which Marx introduced his mature political-economic views (published long-form in the first volume of \textit{Das Kapital} two years later), which was published posthumously in the form of a pamphlet entitled \textit{Value, Price, and Profit}. }---we focus on four such episodes that were particularly impactful, we argue, with respect to the development of Marx's thought. We begin with his first major ``public break'' (his break with the liberalism of his father and schoolmasters in Trier, via his turn to Hegel in the late 1830s, only being known to us by way of his private letter to his father cited previously), his polemics against the pure philosophizing of the Young Hegelians between 1843 and 1845, culminating in his justly famous ``Theses on Feuerbach'' which asserted his commitment to a philosophy of \textit{praxis} -- a commitment to \textit{engaged} philosophy which aimed not only to understand society but also to change it.

Next we move to the period from 1845 to 1849, characterized by the buildup to and eventual defeat of the 1848 revolutions which swept across Europe. We focus especially on his critique of ``state socialists'' like Louis Blanc here, since this became a central focal point around which socialist polemics swirled after Blanc's appointment to the revolutionary provisional government in 1848, where he was expected to begin implementing his scheme for ``national workshops'' as outlined in his tract \textit{Organization of Labor} (1840)\footnote{In reality, Blanc was almost completely hamstrung in these efforts from the start, and almost surely doomed to failure. While public perception, shrewdly encouraged by his opponents, was that he had been tasked with implementing the workshops, in fact he had only been appointed to head the ``Luxembourg Commission'' where he battled with his cynical rival representatives just to complete a report on the feasibility of his schemes, thus keeping him preoccupied while forces of reaction and monarchical restoration worked to defeat the gains of the revolution outside of these meetings. See, e.g., \cite{agulhon_republican_1983}.}