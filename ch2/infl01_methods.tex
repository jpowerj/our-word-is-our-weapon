The methodological contribution of this paper is twofold: first, we leverage insights from computational stylometry to study how Marx's \textit{rhetorical style} drew on (or deviated from) the rhetorical strategies of his ideological predecessors. Then, we utilize state-of-the-art neural contextual embedding techniques to trace the trajectory of Marx's thought over time, relative to European socialist thought in general, through ``semantic space'' -- i.e., a geometric space constructed to map out the language of the different critiques of capitalism included in our corpus, from moralistic to Hegelian to political-economic.

\subsubsection{Stylistic Similarity: Computational Stylometrics}

\subsubsection{Semantic Trajectory: Contextual Embeddings via Neural Transformers}