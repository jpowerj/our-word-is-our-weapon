The key elements of Karl Marx's critique of capitalism was forged over the course of a decade -- the 1840s -- in which a whirlwind of revolutionary upheavals and governmental expulsions swept him from Germany to France, Belgium, and England, where he settled after the abortive revolutions of 1848. Historians thus commonly agree that his thought represents a mixture of German philosophy, French socialism, and British political economy, but disagreements begin to arise when the details of this mixture are interrogated: What particular concepts did he absorb from each, and to what extent did he modify or transform them? Did the influence occur gradually, through e.g. his day-to-day interactions with workers in Paris? Or can we pinpoint particular moments when his reading of certain texts immediately affected his thought?

In this paper we show how one can utilize a set of techniques from Natural Language Processing (NLP) to decompose an author's writings (and the writings of their influences and interlocutors) into semantic and syntactic components, thus \textit{embedding} them in a semantic space and a syntactic space, spaces within which these texts occupy particular points, in the geometric sense. The semantic-analysis algorithm is designed to map texts into points in the semantic space such that texts discussing similar topics, concepts, terms, etc. will be close to one another. The syntactic-analysis algorithm, on the other hand, is designed to capture a text's rhetorical mode of presentation -- for example, terseness or complexity of word choices -- and map rhetorically-similar texts close together.

Thus, by running these algorithms on two or more texts and analyzing how close or distant their semantic and syntactic embeddings are, or how subsets of the texts cluster or fail to cluster within the space, we can begin to trace out an author's evolution in terms of their trajectory through this space over time: for example, a key hypothesis derived from the literature on the evolution of Marx's thought would be that texts from the ``early Marx'' period (roughly from his first writings to the ``Economic and Philosophic Manuscripts of 1844'') would remain within the vicinity of a cluster of Hegelian texts -- whether Hegel's own texts or the texts Young Hegelians like Bruno Bauer -- but afterwards would rapidly move away from the Hegelian discourse and rhetoric cluster and towards clusters formed by French socialist republicans (Louis Blanc, Pierre-Joseph Proudhon, Charles Fourier, Sismondi) and, eventually, British political economists (Adam Smith, David Ricardo, J. R. McCulloch).

With these measures of semantic and stylistic similarity in place, then, we can track the influence exerted by various authors and texts upon Marx over time (Part I) and the subsequent influence of Marx upon 19th-century European socialist discourse. To this end, another key contribution of this work is the release of a series of datasets, the \textit{Digital Marxism Collection}, through which Marx's reading and note-taking, writing, and referencing of other texts can (in almost all cases) be traced down to the exact day. The three main datasets containing this information are as follows:

\begin{itemize}
    \item The \href{https://airtable.com/shrFKP7fp64su32Ad}{\textit{Digital Marx-Engels Register}} (hereafter shortened as \textit{Digital Register}), containing a record for every known piece of writing by the two authors ($N > 1000$), linked to the full text of the writing along with a substantial set of metadata: dates of writing and first publication (cross-referenced with the \textit{Digital Chronicle}--see below), original language, listing of known translations, the location of the work within \textit{MEGA}¹, \textit{MEGA}², \textit{MEW}, and \textit{MECW}, and all known details of the relative contributions of Marx and Engels in the case of jointly-authored works. This dataset extends the original \textit{Marx-Engels Register} \citep{draper_marx-engels_1985a}, which itself built upon and extended a large collection of previously-published bibliographies\footnote{The most comprehensive of these being Maximilien Rubel's 1960 \textit{Bibliographie des oeuvres de Karl Marx} \citep{rubel_les_1960}.}. The key contribution of this dataset, apart from the digitization and relational-database-structuring of the out-of-print 1985 \textit{Register}, is the inclusion of references to the volumes and pages of the ``New \textit{MEGA}'', \textit{MEGA}². While only 16 volumes had been published by mid-1984, when the original \textit{Register} was finalized\footnote{These published volumes being: I/1, I/10, I/22, I/24, II/1, II/2, II/3, II/5, III/1, III/2, III/3, III/4, IV/1, IV/2, IV/6, and IV/7. See \cite{draper_marx-engels_1985a} pp. 206--207 for information on the progress of \textit{MEGA}² publication at the time of compilation.}, an additional 56 volumes have been published in the subsequent years, bringing the total to 72 out of a planned 122. Thus, the \textit{Digital Register} brings the original references up-to-date with these subsequently-published volumes.
    
    \item The \href{https://airtable.com/shr54mI2zLUNzPQUB}{\textit{Digital Marx-Engels Notebooks}} (hereafter shortened as \textit{Digital Notebooks}), containing a record for every entry--i.e., every referenced work--across Marx's 168 extant reading notebooks\footnote{A listing of the contents of these notebooks, along with facsimiles of each hand-written page, are available via the IISG website at \href{https://search.iisg.amsterdam/Record/ARCH00860/ArchiveContentList\#A072e534c62}{https://search.iisg.amsterdam/Record/ARCH00860/ArchiveContentList\#A072e534c62}}, along with (a) metadata like the number of pages and number of excerpts taken from each referenced work, and (b) links to a supplementary dataset containing metadata about the referenced works, along with the full text of the works (when available).
    
    \item The \href{https://airtable.com/shrp3Al2yPCxmoG15}{\textit{Digital Marx-Engels Chronicle}} (hereafter shortened to \textit{Digital Chronicle}), containing a record for each known ``event'' in Marx's and Engels' lives\footnote{See \cite{draper_marx-engels_1985a} for what types of events qualify for inclusion.}, cross-referenced with entries in the \textit{Digital Register} and \textit{Notebook Collection}.
    
    \item The \href{https://airtable.com/shrAJhrqLk8y3qySU}{\textit{Digital Marx-Engels Correspondence}} (hereafter shortened to \textit{Digital Correspondence}), containing a record for each known letter send by or to Marx and/or Engels, along with (a) metadata like date sent and date of receipt, (b) full text of the letter (where available), and (c) links to entries in the \textit{Digital Glossary} (see below) for each author/recipient.
    
    \item The \href{https://airtable.com/shrVoS77J9BOJiDbF}{\textit{Digital Marx-Engels Glossary}} (hereafter shortened to \textit{Digital Glossary}), containing a record for every ``entity''---people, places, organizations, publications, and geographic/geopolitical entities\footnote{See Appendix \ref{app:datasets} for details regarding what types of entities qualify for inclusion.}---referenced across the \textit{Digital Chronicle} and \textit{Digital Correspondence}, along with metadata like links to the Wikidata database id for each entry (which, in many cases, is subsequently linked to the same entity in hundreds of additional databases---see, e.g., the entry for Karl Marx here: \href{https://www.wikidata.org/wiki/Q9061}{https://www.wikidata.org/wiki/Q9061}), where available. As with the \textit{Digital Chronicle} and \textit{Digital Register}, this dataset builds upon and updates the original 1985 \textit{Glossary} with references to the subsequently-published volumes of \textit{MEGA}² and the newly-referenced entities therein.
\end{itemize}

%We begin with a discussion of the postulated sources of influence in Section \ref{sec:influence}, covering both (a) what the literature has to say about which authors had a particularly strong influence, and then (b) examining what particular works of these authors Marx had read up at various points in his publication history. In Section \ref{sec:methods} we describe the semantic and syntactic similarity algorithms utilized in the study, and argue that they allow us to capture notions of influence that can be brought to bear on debates within the literature regarding the comparative magnitude of Marx's influences. Next, in Section \ref{sec:results}, we present the findings from the computational text analysis: the quantitative similarity scores in terms of rhetorical styles (i.e., between the works of Hegel that Marx had read up to a period of writing, and the result of that period in the form of the produced text) as well as a separate set of similarity scores based upon the semantic content of what concepts receive more or less attention. Finally, in Section \ref{sec:conclusion} we discuss the implications of these findings in terms of which hypotheses in the history of political thought literature they support or fail to support, and conclude with a discussion of the broader contribution of the work, namely, a toolkit which can be used as one input to the adjudication of controversies and debates within political theory and the history of political thought writ large.