What do political thinkers \textit{do} with their words when they perform a political speech act -- a book, a pamphlet, a speech, and so on? Since the 1960s, ``Cambridge School'' historians of political thought like Quentin Skinner and J. G. A. Pocock have developed novel understandings of several key historical thinkers and texts, by employing a linguistic-philosophical and context-sensitive approach to this question (\cite{skinner_meaning_1969}, \cite{pocock_virtue_1985}). Drawing on the philosophy of J. L. Austin and the late Wittgenstein, and the structuralism of Ferdinand de Saussure, these scholars have shifted our historiographic focus away from the notion of ``perennial questions'' in political thought \parencite{bevir_are_1994}, and towards a conception of historical texts as \textit{interventions} into a particular, localized discourse.

Roughly speaking, an earlier school of historians of political thought, associated with Leo Strauss, viewed the ``great minds'' of history -- e.g., Plato and Aristotle, Machiavelli and Hobbes, Kant and Hegel -- as engaged in a collective conversation on the ``eternal questions'' of philosophy (in particular, the question of what constitutes a good life and a good society). Importantly, in Strauss' view, it is only this echelon of great minds who qualify as true political philosophers, since their philosophical work is ``animated by a moral impulse, the love of truth'', not the desire to win an argument or persuade the public to adopt their preferences \parencite{strauss_what_1959}. In fact, under this conception, political philosophers must divorce themselves entirely from local or day-to-day political concerns, as ``it is only when the Here and Now ceases to be the center of reference that a philosophic or scientific approach to politics can emerge''.

A Cambridge School approach, on the other hand, asserts that no such separation from the day-to-day political issues of a thinker's time is possible. The claim is stated in its most direct form in Skinner's ``Meaning and Understanding in the History of Ideas'':

\begin{quote}
There simply are no perennial problems in philosophy: there are only individual answers to individual questions, and as many different questions as there are questioners. There is in consequence simply no hope of seeking the point of studying the history of ideas in the attempt to learn directly from the classic authors by focusing on their attempted answers to supposedly timeless questions.
\end{quote}

For instance, to take an obvious example, the Cambridge School would view Machiavelli's \textit{The Prince} not as uninterested philosophical reflections on just rule and the structure of a just society, but instead as aiming to \textit{do} something, to accomplish some desired end -- in this case, to convince the new Medici regime to employ Machiavelli as a political advisor, after he had lost his patronage due to the fall of the previous regime. Or, to take a slightly more controversial example\footnote{``Controversial'' in the sense that practitioners of the Cambridge School approach, along with Wood herself, view the approach of this work as differing in some respects from the ``orthodox'' Cambridge School approach established in the works of e.g. Skinner, Pocock, and John Dunn.}, Ellen Meiksins Wood's \textit{Citizens to Lords} challenges Strauss' conception of Plato's \textit{Republic} as an uninterested, non-partisan tract on perennial questions of the good society, and instead argues that Plato's aim was to justify the continuation of the dictatorial rule of the 40, which had just come to power and restored stability in Athens, terminating Plato's enslavement.

