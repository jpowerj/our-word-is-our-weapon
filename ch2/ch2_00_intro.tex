How important were the preexisting ideas and concepts of German philosophy, French socialism, and British political economy to the formation of Karl Marx's thought over the course of the 1840s? And, once he had articulated his resulting critique of capitalism, how  was he able to cement its concepts and vocabulary as the hegemonic discursive frame utilized by socialist movements across Europe, as opposed to those propounded by other popular socialist writers of the era like Pierre-Joseph Proudhon, Louis Blanc, Ferdinand Lassalle, and Mikhail Bakunin? What was he ``doing'' when he intervened in the socialist discourse of the time via public speech acts---books, articles, speeches, and so on---and how successful was he in this endeavor? We operationalize these questions in quantitative terms, as empirically testable hypotheses, and then leverage recent advances in neural network-based text analysis methods to attempt answers to them in an empirical rather than purely speculative or interpretive mode.

In the remainder of the Introduction, we start in Section \ref{intro:hpt} with an overview of the Cambridge School approach to the study of historical political thought, discussing how our methods herein serve to instantiate some of the historiographic insights developed by this strand of research. We then present a high-level outline of the development of Marx's thought in Section \ref{intro:marx}, omitting details which will be explored in-depth in Parts I and II of the work, and end the Introduction with a similarly broad introduction to the computational methods employed in the remainder of the work, in Section \ref{intro:methods}. 

In Part I we carry out a computational decomposition of the texts thought to have influenced Marx's thought, evaluating their relative importance to its genesis and evolution from his 1836 poetry to his work on the three volumes of \textit{Capital} which occupied him from the late 1850s until his death in 1883. We propose a method for identifying the contours of a ``discursive field'', given a set of representative texts from this field, by identifying clusters of terms which are most unique to this particular discourse relative to a larger semantic embedding space (in this case, a space capturing the language of 19th-century literature writ large). We then use this method to operationalize the notion of how ``Hegelian'', ``socialist'', or ``political-economic'' an author's writings are, and find (lending credence to what is commonly assumed or asserted in studies of nineteenth-century radical political thought) that (a) Marx's writing becomes less ``Hegelian'' and more ``socialist'' as he is exposed to Paris' brand of working-class-oriented socialism between 1843 and 1845, and then (b) becomes less ``socialist'' and more focused on issues of political economy over the remainder of his life in London, from 1849 onwards.

In Part II we turn from an analysis of the earlier influences \textit{on} Marx to an analysis of Marx's subsequent influence on the political discourse of his own time -- in particular, his influence on the rhetoric and practice of the European socialist movement from the 1850s onward. We ask: what did he \textit{do} with the concepts bequeathed to him by his influences, those analyzed in Part I? How did he hope to ``steer'' socialist discourse via his interventions, and how successful was he in these various attempts? We again construct an embedding space, this time ``tuned'' to represent authorial position-taking within European socialist polemics of the time, and argue that the movement of Marx's discourse relative to this general socialist discourse strongly supports the perspective that (a) his aim was to ``pull'' socialists away from moralistic discourse and towards a more positivistic political-economic discourse, and that (b) he was successful in this endeavor, as evidenced by the similarity between the trajectory of Marx's thought and the subsequent trajectory of general socialist thought within this ideological (moralistic vs. political-economic) space.
