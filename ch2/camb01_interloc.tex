\subsubsection{The Young Hegelians, 1843--1845: Bruno Bauer and Max Stirner\label{sec:hegelians}}

Although he had planned to write critiques of particular contemporary authors like Hermes or Rosenkranz as early as 1839, it was his critique of Hegel's Philosophy of right in 1843 that scholars typically point to as marking the beginning of Marx's period of ``committed'' polemics against the Young Hegelian milieu which he had initially identified himself as part of. After the Critique of Hegel's Philosophy of Right, his first true engagement with a Hegelian ``interlocutor'' (as opposed to Hegel himself, who had been dead for 12 years when Marx wrote the Critique of his Philosophy of Right) came in the form of his writing ``On the Jewish Question''. The themes of this work were further expanded upon, and the scope of his polemics expanded to include Max Stirner and Ludwig Feuerbach, in his first joint work with Engels in 1844, \textit{The Holy Family}. Marx considered this work to be his ``final break'' with the Young Hegelians, with the exception of Feuerbach -- his final break with the latter would not come until his second joint work with Engels, the \textit{German Ideology}, written in 1846 but never published in Marx's or Engels' lifetime (in fact, not published until the 1930s).

``the sovereign
derision that we accord to the Allsemeine Literatur-Zeituns is in stark
contrast to the considerable number of pages that we devote to its
criticism'.'' Engels to Marx

\subsubsection{``State Socialism'' I, 1845--1849: Louis Blanc\label{sec:blanc}}

Although a number of works have explored Marx's views on the State in detail (e.g., \cite{hunt_political_1974} and \cite{hunt_political_1984}; \cite{draper_karl_1977-1}; and \cite{leipold_radical_2020}, Ch. 7), these investigations typically focus on Marx's work after the rise and fall of the Paris Commune in 1871, especially his 1875 \textit{Critique of the Gotha Programme}. In this section, however, we hope to draw more attention to Marx's (admittedly more opaque and imprecise) work from an earlier era, namely, the years leading up to the 1848 revolutions. In his writings from this period one can already infer the main characteristics of Marx's burgeoning socialist conception of the state, especially in his criticisms of European ``state socialists'' like Louis Blanc who envisioned a non-revolutionary path to socialism set into motion by the introduction of universal suffrage and constitutional constraints on power.

\subsubsection{Anarchism I, 1846--1849: Pierre-Joseph Proudhon\label{sec:proudhon}}

Pierre-Joseph Proudhon's thought, as our quantitative analysis will corroborate, 
%was the result of 
was an eclectic and self-taught mixture of his 1830s engagement with theology and philology in the 1830s with his subsequent engagement---predating Marx's own by a few years---with British political economy and Hegelian philosophy. In fact, although details on their engagement are scarce\footnote{Scarce despite numerous research endeavors, typically in the same anarchists-versus-Marxists vein as the studies on Marx and Bakunin described in more detail in Section \ref{sec:bakunin} below).}, some researchers accept Marx's post-falling-out contention that he taught Proudhon everything he knew about Hegel.

In fact, although most works on the early development of Marx's thought contend that Marx's transition from Young Hegelianism to British political economy was driven by Engels' 1843 engagement with the latter, published in his ``Umrisse'', a good case could be made for the hypothesis that Proudhon played a not-insignificant role. As a comparison of their reading notebooks (Table \ref{tab:reading}) attests, Proudhon had already read many of the texts noted as central to this Engels-to-Marx transmission three or four years before Marx began studying them.

\begin{table}[ht!]
    %\begin{minipage}{\linewidth}
    \centering
    \begin{threeparttable}
    \begin{tabularx}{\linewidth}{Xcc}\toprule
        \textbf{Text} & \textbf{Proudhon} & \textbf{Marx} \\\midrule
        
        Adam Smith, \textit{Wealth of Nations} (1776) & Oct 1841 & Mar--Aug 1844 \\
        % cahier XXIII (oct 1841) (6-12), p. 1090
        % again, cahier premier de 1844, jun-july (ed. Blanqui) (18-45), p. 1092
        % First mentioned in comments on james mill
        
        David Ricardo, \textit{Principles} (1817) & Oct 1841 & 1844 \\\midrule
        % cahier XXIII (oct 1841) (42-48), p. 1090
        % First referenced in econ+phil manus of 1844
        
        Charles Comte, \textit{Traité de la propriété} (1834) & 1839 & --- \\
        % cahier I in-4, p. 1081
        % also, cahier X (1839) (p. 15-20), p. 1086
        
        F. X. J. Droz, \textit{Propriété} (1832) & 1839 & 1846 \\
        % cahier I in-4, p. 1081
        
        A. Destutt de Tracy, \textit{Économie politique} (1823) & 1839 & Oct 1843-1845 \\
        % cahier II in-4 (p. 17-29), p. 1081
        
        Adolphe Blanqui, \textit{Hist. de l'econ. pol.} (1837) & 1839 & 1845 \\
        % cahier VI in-4 (p. 15), p. 1083
        
        Dugald Stewart, \textit{Esquisses de phil. morale} (1793) & 1839 & 1858--1862 \\
        % cahier III (1839) (p. 19-22), p. 1084
        
        J. B. Say, \textit{Cours complet de Écon. pol.} (1828) & 1839 & Oct 1843--1845 \\
        % cahier VII (1839) (p. 29-48), p. 1086
        
        A. A. Cournot, \textit{Principes mathém. d'Écon. pol.} (1838) & 1839 & --- \\
        % cahier X (1839) (p. 27-28), p. 1086
        
        P. Rossi, \textit{Cours d'Écon. pol.} (1836) & Jan 1840 & 1845 \\
        %  cahier XI (6 jan 1840) (3-4), p. 1087
        % again cahier XXIII (oct 1841) (book 1) (12-42), p. 1090
        % again cahier XXIV (nov 1841) (book 2) (21-38), p. 1090
        
        G. Garnier, \textit{Da la propriété} (1792) & Nov 1840 & ---
        % cahier XV (nov 1840) (35-36), p. 1088
        
        % Marx did, in 1863, read Garnier's 1796 S. 132-145: Germain Garnier , Abrégé élémentaire des principes de l'économie politique , 1796. IISH B103
        \\
        
        %Buchez & 1840 & & \\
        
        %Bentham & 1841 & & \\
        
        %C. Fourier, \textit{Traité d'association} (1822) & Sep 1841 & --- & --- \\
        % Engels read Charles Fourier , Nouveau Monde Industriel et Sociétaire , ab ± 1877?, franz. ½ S
        % S. 8-10: Ch. Fourier , Theorie des 4. Mouvements , 1846.
        
        %Considérant, \textit{Politique générale} & Sep 1841 & & \\
        
        %Cabet, \textit{Icarie} & Sep 1841 & & --- \\
        
        A. Ciezkowski, \textit{Du crédit et de la circulation} (1839) & Oct 1841 & --- \\
        % cahier XXII (oct 1841) (3-34), p. 1090
        
        %P. Leroux, \textit{l'Humanité} (1840) & Jan 1842 & & \\
        
        %Buret, \textit{De la misère} (1840) & 1844 & & \\
        
        %G. Garnier, preface to Smith & Jul 1844 & & \\
        
        %M. Chevalier, Cours d'Econ. pol.
        % Le premier de 1844, (1841-1842) (15-18), p. 1091
        
        %Sismondi, \textit{Études sur l'Écon. pol.} (1837-38) & Jul 1844 & & 1845\tnote{z} \\
        % le second cahier, jul 1844, etudes (book 2) (1838) (19-38), p. 1092
        % B30 S. 7--19 (Brussels)
        
        %W. Godwin, \textit{Recherches} & Jul 1844 & & \\
        
        %L. Blanc, \textit{Histoire de dix ans} & Jan-May 1844 & & ---\tnote{z} \\
        \bottomrule
    \end{tabularx}
    \end{threeparttable}
    %\footnotetext{\cite{rubel_marx_1975} claims, however, that Marx ``presumably read'' Ciezkowski's \textit{Historiosophie}, however.}
    %\end{minipage}
    \caption{A comparison of the dates of first reading for key political-economic texts, as recorded in Proudhon's and Marx's respective reading notebooks. On Proudhon's reading notebooks, see Appendix \ref{app:proudhoncarnets}. On Marx's, see Appendix \ref{app:marxreading}. Entries after Smith and Ricardo are listed in the order in which they appear in \cite{haubtmann_pierre_1982}. Sources for each date of reading are given in Appendix \ref{app:tablefn}.}
    \label{tab:reading}
\end{table}

\subsubsection{Challengers to the Throne I, 1859--1860: Karl Vogt\label{sec:vogt}}

As chronicled by several of his biographers, Marx's political-economic writing incurred several major interruptions in the form of protracted polemics against other socialists whom Marx viewed as potential opponents (or perhaps saboteurs) for hegemony over European socialist discourse. In 1859 and 1860 for example, after the publication of his \textit{Zur Kritik}, Marx abruptly ceased working on his ``Economics'' and began a polemic with Karl Vogt, a minor figure from the 1848 Frankfurt Parliament who had slandered him in a German newspaper. The extent to which his need to strike back suddenly superceded all other concerns is aptly described by David McLellan in his biography of Marx:
\begin{quote}
This quarrel, which occupied Marx for eighteen months, is a striking example both of Marx's ability to expend tremendous labour on essentially trivial matters and also of his talent for vituperation. (\cite{mclellan_karl_1973}, p. 311)
\end{quote}
Even Engels himself, normally supportive of Marx in all his endeavors, diplomatically begged the latter not to allow the ``Vogt affair'' to interrupt his political-economic studies:
\begin{quote}
The prompt appearance of your second installment\footnote{Referring to the ``sequel'' to Marx's 1859 \textit{Contribution to the Critique of Political Economy} (\textit{Zur Kritik der Politischen Ökonomie}), i.e., to the work that would eventually coalesce into the three volumes of \textit{Das Kapital}.} is obviously of paramount importance in this connection and I hope that you won't let the Vogt affair stop you from getting on with it. [...] I am very well aware of all the other interruptions that crop up, but I also know that the
delay is due mainly to your own scruples.
\end{quote}
Marx did not heed Engels' plea, however, and pressed onwards with his 18 months of work on what was to become the 208-page \textit{Herr Vogt}. Characteristically, however, Marx was never able to find a German publisher, thus defeating the entire purpose of the work in the first place.

% 473 After the publication, in June 1859, of the first instalment of A Contribution to the Critique of Political Economy (see present edition, Vol. 30), Marx intended, as previously agreed with the Berlin publisher Duncker, to prepare for the press and publish as the second instalment the 'Chapter on Capital', which constitutes the bulk of his main economic manuscript of 1857-58; and then publish the remaining parts of his economic work (see Notes 250 and 355).

% As he proceeded with his plan, however, he realised that he would have to do more research to formulate the basic propositions of his economic theory. But his journalistic activity and other party obligations, above all the need to refute publicly Vogt's slanderous allegations against proletarian revolutionaries, temporarily diverted him from his economic studies. It was not until 1861 that he resumed them in earnest. Later Marx decided to publish his researches not as the second and further instalments of A Contribution to the Critique of Political Economy but as a large independent work.—489, 498, 502, 508, 511, 522, 523, 542, 574

\subsubsection{``State Socialism'' II, 1864--1883: Ferdinand Lassalle\label{sec:lassalle}}

Ferdinand Lassalle, a German socialist intellectual and agitator\footnote{The rendering of his surname as ``Lassalle'' is actually a Gallicization of his family name, Lassal, a spelling he promulgated early on to deflect attention away from his Silesian origins, as part of his goal to establish himself as a radical intellectual in Paris starting in the mid-1840s.} played a variety of seemingly-incongruous roles in the development of Marx's post-1848 social and economic thought. A study of \textit{MEGA}², for example, would give an impression of Lassalle as someone with whom Marx tried to remain on good terms via his direct correspondence, despite harboring an intense disdain towards him (which comes out in his descriptions of Lassalle in letters to Engels over the same time period), a balancing act which was ``resolved'' by Lassalle's early death after an 1864 duel. If one takes into account late-19th-century developments in European socialism immediately before and after the 1875 establishment of the German SPD, however, it emerges that in fact Lassalle's immense posthumous influence outlived even Marx himself. It wasn't until Engels' work moulding the ideology of the SPD for 12 years following Marx's death that Lassalleanism was ``defeated'' as a viable competitor to Marxism among European socialists.


\subsubsection{Anarchism II, 1872--1883: Mikhail Bakunin\label{sec:bakunin}}

Mikhail Bakunin, hailed by some as the father of modern anarchism, is typically cast in the role of Marx's main rival in the First International between 1868 (the year Bakunin joined) to 1872 (when Bakunin and his followers were expelled by Marx), in socialist and anarchist histories alike (see, e.g., \cite{eckhardt_first_2016}). Interestingly, however, Marx and Bakunin crossed paths fairly regularly, in substantial ways, from 1840 onwards. To name just one rarely-mentioned instance, Bakunin produced the first Russian translation of the \textit{Communist Manifesto}, which was published in the periodical \textit{Kolokol} in London in 1860\footnote{See \cite{guillaume_internationale_1905}, p. 283, cited in \cite{favilli_history_1996}. Bakunin had a number of path-crossings with Engels over the years, as well. They were both in attendance, for example, at F. W. J. Schelling's infamous 1841 lectures at the University of Berlin (\cite{hunt_marxs_2010}, 44--46).}. Starting the narrative of the Marx-Bakunin relationship in 1864, therefore, ignores a great number of interactions which impacted the development of Marx's thought.

Bakunin moved from Moscow to Berlin in 1840 to enroll at the University of Berlin---the same university where Marx had been studying since 1836---and quickly became a prominent figure in the Young Hegelian movement alongside Marx, Bruno and Edgar Bauer, and Arnold Ruge. Bakunin and Marx both, in fact, contributed articles to Ruge's \textit{Deutsche Jahrbücher fur Wissenschaft and Kunst} in 1842, though Marx's contribution (ironically, a commentary on Prussian censorship restrictions) was censored by the Prussian government and only published a year later in Switzerland. After the Prussian government banned this publication outright in 1843, Marx and Ruge moved to Paris to co-found the \textit{Deutsche-Franzosische Jahrbücher}, with Bakunin joining them in the city that same year. After finally meeting in person in 1844, Marx and Bakunin corresponded in a mostly-cordial fashion for decades, up until the 1872 split of the First International. Even as late as 1871, for example, Bakunin accepted a commission to produce the first Russian translation of Volume 1 of \textit{Das Kapital}, a work which he deeply admired, having earlier commented that ``no other work that I know of puts together such a profound, enlightening, scientific, decisive analysis'' of the capitalist economy.

The 1872 split and the four years leading up to it---an episode of Marx's life which Alvin W. Gouldner calls ``the culminating conflict of [Marx's] political life'' (\cite{gouldner_marxs_1982})---have been exhaustively documented in two parallel literatures, which present two starkly contrasting narratives. The first narrative, promulgated most heavily in the Soviet Union, sees Marx effortlessly fusing theoretical insight with organizational prowess, keeping the International sharply focused on its proletarian revolutionary aims despite the best efforts of the saboteur Bakunin\footnote{See \cite{nicolaievsky_karl_1936}, pp. 280--297, for a fairly inocuous example.}. The second narrative, promulgated by both Western anti-Soviet historians and anarchists in nearly identical forms, sees Marx ruthlessly stamping out any and all anti-authoritarian voices in the International, with Bakunin finally giving up on his noble but quixotic efforts in 1872 to found the aptly-named Anti-Authoritarian International\footnote{A stark example of this contrasting narrative can be found in \cite{eckhardt_first_2016}.}.
%(from which a direct line can be traced to the present-day IWA-AIT\footnote{Short for the International Workers' Association---Asociación Internacional de los Trabajadores, whose Spanish section constitutes the major Spanish trade union CNT (Confederación Nacional del Trabajo).})

For the purposes of this work, however, it suffices to say that Marx viewed Bakunin as a key rival for leadership of the European socialist movement. Public perception and commentary on this movement, especially in the years leading up to the split, often compared the two, adding fuel to Marx's competitive fire. The Italian socialist newspaper \textit{La Plebe}, for example, characteristically referred to Marx as ``Germany's Bakunin'' in a major article of January 1872\footnote{``Lettere da Berlino'', \textit{La Plebe}, Jan. 5, 1872, cited in \cite{favilli_history_1996}, p. 32. See \textit{ibid.} pp. 20--46 for an in-depth discussion of the Bakunin-Marx rivalry and its relation to 19th-century Italian socialist thought.}. Hence, as is the case with nearly all of his works, Marx's discursive interventions throughout the era of the First International were driven primarily by polemical concerns. 

Just as it is inappropriate to begin the Marx-Bakunin narrative in 1868, it is also inappropriate to end it with the 1872 split. Two years after the split, in April of 1874, Marx began reading Bakunin's \textit{Staatlichkeit und Anarchie} (\textit{Statism and Anarchy}). By the time he finished in January of 1875, he had copied 224 separate extracts into his notebook in Russian, some spanning several pages. He provided extensive commentary on 39 of these, breaking out of the extracts and writing paragraph-length or even page-length responses, in addition to the shorter inline comments he made on nearly all of them (ranging from single exclamation points to parenthetical definitions, translations, and quips)\footnote{Our calculations, based on \mew{18}, pp. 597--642.}.



\subsubsection{Challengers to the Throne II, 1883--1884: Eugen Dühring\label{sec:duhring}}

When in the post-\textit{Kapital V1} years another contender for Marx's historiographic crown emerged, Eugen Dühring, Engels (having learned from the Karl Vogt episode) consciously opted to take the lead and conduct the polemics himself so Marx could carry on with his work on \textit{Kapital}. Although Marx did end up contributing in a non-trivial way to the resulting book \textit{Herr Eugen Dühring's Revolution in Science} (typically shortened as \textit{Anti-Dühring}), it was published under Engels' name in 1884, a year after Marx's death.

Unlike in the case of Vogt, however, Dühring was a worthy opponent, a major intellectual figure in Germany who wielded great influence and thus directly challenged the recently-acquired gains in Marx's prominence and notoriety after the rise and fall of the Paris Commune. Dühring's published works would have an immediate and substantial impact on German political-philosophical discourse---albeit an impact fairly distant from the epicenters of socialist discourse---with Friedrich Nietzsche being only one of many prominent post-Hegelian thinkers who were profoundly influenced by Dühring\footnote{Nietzsche's reading included nearly all of Dühring's published works, some of which he read on multiple occasions---see the supplemental dataset on Nietzsche's known and conjectured reading described and linked in Appendix \ref{app:suppdata}. For a summary of the main trends in post-Hegelian German philosophy, see \cite{beiser_after_2014}, pp. 172--184 (``Dühring on the Value of Life''), where Dühring's ``important place in the history of nineteenth-century philosophy'' includes his role as ``the founder of  German positivism, the grandfather of Schlick, Carnap, Neurath, and Reichenbach.'' (p. 174)}.
