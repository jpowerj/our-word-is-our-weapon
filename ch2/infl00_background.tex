Lenin's aphorism is perhaps the most oft-cited assertion that Marx's thought developed at the intersection of German philosophy, French socialism, and British political economy, but it is certainly not the earliest. Marx himself emphasized the necessity of combining these regional strands of thought as early as 1844 in an article for \textit{Vorwärts!}, stating that ``the German proletariat is the theoretician of the European proletariat, just as the English proletariat is its economist, and the French its politician.''\footnote{Marx, ``Critical Marginal Notes on the Article `The King of Prussia and Social Reform. By a Prussian'{}'', \textit{Vorwärts!}, No. 64, August 10, 1844. \mecw{3}, p. 202.} In fact, several of Marx's precursors had voiced a similar sentiment regarding the necessity of trans-national philosophical and practical development: Ludwig Feuerbach had earlier proclaimed that ``the `new' philosophy, if it wished to be at all effective, would have to combine a German head with a French heart,''\footnote{\cite{mclellan_karl_1973}, p. 64.} while Moses Mess, an early convert to Communism (during his time in Paris) who worked with Marx and Engels in Cologne as part of the \textit{Rheinische Zeitung} in 1843, had published a book in 1841 titled \textit{Die europäische Triarchie} (\textit{The European Triarchy}) \parencite{hess_europaische_1841} emphasizing the necessity of precisely the tri-national combination under consideration here (i.e., thus adding British political economy into the mix). It is with these considerations in mind that we now survey the subsequent literature on the respective influences of German philosophy, French republican socialism, and British political economy on the development of Marx's thought.
