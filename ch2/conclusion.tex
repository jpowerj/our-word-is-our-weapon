Modern scholarship on the history of political thought, and especially the questions of intellectual influence and the locutionary impact of a thinker's interventions with respect to a broader discourse, have given rise to a rich body of work stressing the importance of formerly non-canonical thinkers---``major'' and ``minor'' interlocutors---for deepening our understanding of a given text. In this work we have shown how this type of context-inclusive analysis, though it typically expands the necessary amount of reading beyond the limits of individual human capabilities, can be brought back into the realm of possibility with the aid of modern computational-linguistic tools. We have also highlighted, however, the crucial domain expertise and interpretive skills which are still required of a researcher or team of researchers in order to carry out such a computer-aided study successfully. This ``outsourcing'' of certain aspects of a study to computational tools can, in fact, have pernicious consequences, drastically
%increasing
amplifying the risk of drawing invalid inferences if hidden assumptions and seemingly-inconsequential methodological choices are not interrogated with a critical eye.

An additional consideration must therefore be kept in mind when determining whether or not to employ the computational methods discussed here in an intellectual-historiographic endeavor, namely, that of the diminishing returns from a wider and wider expansion of the set of texts being analyzed. Although these methods can enhance the replicability and verifiability of a given study, the key contribution which we have emphasized is the ability for these methods to vastly expand the contextual scope of a study---i.e., how many additional texts are taken into account when deriving our conclusions about a given text or discourse of interest. It may in fact be the case, however, that a given author or text \textit{did} intervene in a discourse which was bounded or insulated enough that it remains amenable to standard non-computationally-aided study.

As Leo Strauss' \textit{Persecution and the Art of Writing} brought to the fore of political-philosophical thought, for instance, an author may purposefully restrict their interlocutors to a select few individuals out of fear of censorship or the potential consequences of their views being publicly revealed. Indeed, in the recently-published first volume of his thoroughgoing biography of Marx himself, Michael Heinrich discusses the impact of Reimarus' \textit{Apologie oder Schutzschrift für die vernünftigen Verehrer Gottes} on Marx's early thought (and on 19th-century German philosophy more broadly), despite the fact that it was only ever read by Reimarus' closest friends and never published in his lifetime. In Cambridge School analyses such as ours where the aim is to ascertain what an author was attempting to do with a given text, instances like these where its ideas become influential despite being written and distributed under persecution present a more challenging case, requiring detailed study of a small set of extant material rather than a wider computational study of a broad public discourse, like ours.

More commonly, e.g. for European texts of the Middle Ages up through the Renaissance, it may actually be the case that a thinker wrote with a small select audience in mind, given the rarity of both literacy and access to printing presses (not to mention the financial resources required to fund a widely-distributed publication).
%\footnote{cf. \cite{becker_multiplex_2020} for the case of Martin Luther,  locutions impacted  despite these low literacy rates in general, already in the early 16th century Luther's locutions impacted a massive audience of mostly interlocutors who were mostly unknown to him.}.
Even into the era of the French Enlightenment, many texts which we now consider to be epochal or canonical were originally only addressed to and read by a small set of close confidantes. Indeed, \cite{goodman_republic_1996} describes in detail how pre-Revolutionary French \textit{salonnières} discussed and debated their ideas in tightly-knit, insular communities wherein ``reading one's manuscripts aloud in salons could be an alternative to publication,'' such that ``there were manuscripts that were read in or circulated through salons and never published, such as Gentil-Bernard's `Art d'Aimer,' which went the rounds for years, and Guibert's `Eloge du Chancelier de l'Hospital'{''} (p. 147; see also \cite{chartier_cultural_1981}). In cases like these with an ostensibly bounded contextual scope, a ``classical'' non-computational approach (such as that of \cite{skinner_foundations_1978a} and \cite{skinner_foundations_1978b}, \cite{pocock_machiavellian_1975}, or \cite{baker_inventing_1990}) may be the better option, in terms of eliminating concerns that the computational tools may miss important contextual details or collapse important distinctions in word usage\footnote{Cf., however, \cite{becker_multiplex_2020}, which illustrates how even the early 16th-century spread of Martin Luther's ideas involved a vast network of interlocutors: correspondents, former students, and theological opponents spread across the entirety of present-day Germany and beyond.}.
