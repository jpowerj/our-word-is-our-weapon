Although tracking the literary output of Marx's interlocutors is generally a much more difficult task than tracking Marx's own\footnote{Ostensibly due to the fact that, unlike Marx, most 19th-century thinkers did not end up having a global superpower collecting and propagating their texts---on this topic, see our discussion of the Soviet ``weaponization'' of Marxism in \cite{jacobs_quantifying_2021}.}, a few collections of key rivals exist which enable us to ``break apart'' the Socialism vector into vectors for particular authors. With these individual-author vectors, we can evaluate which authors in particular had trajectories which drove the overall shift observed in the previous section.

Pierre-Joseph Proudhon's writings are of special interest with respect to the computational nature of this work, since not only do we have his collected writings but also newly-digitized scans of his notebooks, the \textit{Carnets}, containing (as in the case of Marx) not only drafts of his works at various stages but also the meticulous reading notes and extracts he kept over the course of his life. An overview of this dataset, used throughout our analysis in Section \ref{sec:marxvproudhon} below, is given in Section \ref{sec:proudhon} above, with more details provided in Appendix \ref{app:proudhoncarnets}\footnote{For an overview of the \textit{Carnets} in general, see the section entitled ``Notes et Annotations Diverses'' in the Appendix of \cite{haubtmann_pierre_1982}, pp. 1079--1098.}.

As for his main body of writings---those which were intended for public consumption---several large collections have been compiled, starting as early as 1850 when he was still actively publishing new works. This first 26-volume collection was completed in 1872, and scans of each volume are available through the Bibliotheque National de France's Gallica portal at \href{https://catalogue.bnf.fr/ark:/12148/cb31154797t}{https://catalogue.bnf.fr/ark:/12148/cb31154797t}.

A comparison of Marx's writings with those of Ferdinand Lassalle, somewhat in contrast to the case of Proudhon, lets us analyze Marx's speech acts as a self-consciously \textit{political} rather than economic actor. While (as argued in Section \ref{sec:proudhon} above) Marx explicitly aimed to distinguish himself as a ``better'' economist than Proudhon, it was Lassalle's political principles and their manifestations in e.g. the Gotha Programme of 1875 (and the Erfurt Programme of 1891, in Engels' posthumous efforts his behalf) that Marx explicitly worked to repudiate and replace with his own.

Due to his prominence in the eyes of Second International-era SPD intellectuals like Eduard Bernstein, new collections of Lassalle's writings were compiled and published quite frequently from 1865 (starting with J. P. Becker's collection published just one year after Lassalle's death) up until the Nazi regime's rise to power. Although some additional collected-works projects were carried out in the GDR from 1949 onwards, three collected-works projects from the SPD era remain basically the canonical reference texts for scholarship on Lassalle to this day. The most commonly-referenced collection, the \textit{Gesammelte Reden und Schriften} (GRS), was edited by Eduard Bernstein and published in 12 volumes from 1919 to 1920, while a second collection, the \textit{Nachgelassene Briefe und Schriften} (NBS), was edited by Gustav Meyer and published in 6 volumes between 1921 and 1925, augmenting the corpus of the earlier 12-volume project with (for example) posthumously-discovered letters and earlier drafts of major speeches found in his notebooks\footnote{Links to each volume are given in Appendix \ref{app:lassallecw}.}. A third collection, Eduard Bernstein's 1898 \textit{Reden und Schriften} (RS) in 3 volumes, is referenced less often but remains influential nonetheless due to its status as the canonical reference for Lassalle's writings from the year of its publication up until the 1920s (when, as the later volume's title suggests, Bernstein's 12-volume GRS supplanted the 3-volume RS)\footnote{It is important to note, however, that (for reasons which are not made entirely clear in Bernstein's introduction) there are some texts in the RS which were \textit{not} carried over into the GRS. For 6 of the 100 texts listed in Andréas' bibliography, therefore, we had to scrape the plaintext by OCRing scans of the original RS, which are of far lower quality than the available scans of the GRS and NBS. As explained in Appendix \ref{app:lassallebib}, however, these 6 texts can be identified and excluded from any analysis by filtering out texts whose \texttt{source} metadata variable is equal to \texttt{"RS"}.}.

More recently, researcher Bert Andréas' valuable bibliography \citep{andreas_ferdinand_1981} contains entries for all known writings of Lassalle, along with known translations and information on differences (e.g., inclusion, exclusion, and modification of the original text) between subsequent editions and printings.

The compilation of our Lassalle dataset, which pairs digital plaintext versions of all his known writings with metadata on each text (e.g., date of writing and/or publication, data on all known versions, on all known translations, etc.), was aided immensely by our digitization of Andréas' bibliography. The resulting diachronic corpus of Lassalle's writings is analyzed in Section \ref{sec:marxvlassalle} below and discussed in detail in Appendix \ref{app:lassallebib}.

Mikhail Bakunin's complete works in German have been published in 3 volumes in an edition titled \textit{Gesammelte Werke} edited by Max Nettlau, available for full viewing at \href{https://catalog.hathitrust.org/Record/012322886}{HathiTrust}.
%A 6-volume selection of works was also published.
In French, there also exists a 6-volume \textit{Oeuvres} available at the \href{https://archive.org/details/oeuvresbs01bakuuoft}{Internet Archive}\footnote{Volume by volume links to this French collection are as follows: \href{https://archive.org/details/oeuvresbs01bakuuoft}{Volume 1}, \href{https://archive.org/details/oeuvresbs02bakuuoft}{Volume 2}, \href{https://archive.org/details/oeuvresbs03bakuuoft}{Volume 3}, \href{https://archive.org/details/oeuvresbs04bakuuoft}{Volume 4}, \href{https://archive.org/details/oeuvresbs05bakuuoft}{Volume 5}, \href{https://archive.org/details/oeuvresbs06bakuuoft}{Volume 6}.}.

However, as can be seen in Figure \ref{fig:bakuninpubs}, which plots Bakunin's published writings over time, 
%(or by quickly scanning the tables of contents of these collections, or reading the secondary literature),
it is only in the period between 1868 and 1872 when Bakunin wrote to any significant degree, with three minor exceptions: the first is his article \textit{Die Reaktion in Deutschland} published in Arnold Ruge's 1842 \textit{Deutsche Jahrbucher} (discussed in Section \ref{sec:bakunin} above), the second an 1848 speech in support of the revolutionary efforts in Poland, and the third his collected correspondence. For example, the 6-volume French collection contains no writings outside of this period, while the \textit{Gesammelte Werke} delves only slightly outside of this range, covering the years 1865 to 1875 (with a single letter to Marx being the only inclusion from 1865, and a total of 20 pages of post-1872 writings). Therefore, although he ran (and published) in the same circles as Marx during the latter's Young Hegelian phase, and although there is some overlap in terms of whom they corresponded with before the era of the First International, the dearth of written material outside of 1868--1872 means we do not have sufficient data for our method to be able to track Marx's influence on Bakunin. There exists, however, a large body of interpretive work in the history of political thought on the mutual influence between the two figures: in addition to the works cited in Section \ref{sec:bakunin} which focus mainly on the era of the First International, \cite{thomas_karl_1980} traces the interaction between Marx and the anarchist movement more broadly over the course of Marx's lifetime, while Marshall S. Schatz's Introduction to \cite{bakunin_statism_1990}---a volume in the Cambridge Texts in the History of Political Thought series, edited by Raymond Geuss and Quentin Skinner---situates Bakunin's thought within the context of both 19th century radical political thought and the political upheavals across Europe during Bakunin's lifetime.

\begin{figure}
    \centering
    \includesvg[width=\textwidth]{bakunin_pubs.svg}
    \caption{Bakunin's Literary Output, 1837--1876}
    \label{fig:bakuninpubs}
\end{figure}


\subsubsection{Marx vs. Proudhon\label{sec:marxvproudhon}}

As we hinted at in Section \ref{sec:proudhon} above, Pierre-Joseph Proudhon's written output is typically characterized as being either brilliantly syncretic or a chaotic jumble of contradictions, depending on the evaluator's tastes and political-theoretic proclivities. In this section we apply the tools we've used throughout this section to evaluate the veracity of these two views, to compare the trajectory of his thought with that of Marx's, and to then draw a set of conjectural hypotheses regarding how these results can shed some light on why Marx's thought ``won out'' over Proudhon's in terms of how strongly they influenced subsequent European socialist discourse.

Works such as \cite{woodcock_proudhon_1956} and \cite{hoffman_revolutionary_1972}, which attempt to organize Proudhon's thought into a coherent set of principles, typically still discuss the challenges inherent in needing to ``de-Hegelianize'' much of his writing. As \cite{hoffman_revolutionary_1972} describes, Proudhon's attempts to apply Hegel's dialectical method to his subject matter often lapsed into exercises in forcing the keywords of the subject to fit into a neat ``thesis-antithesis-synthesis'' equation.

Marx, for example, attacks Proudhon on precisely this point in his letter to Pavel Annenkov criticizing the former's \textit{Systeme des contradictions économiques}\footnote{The contents of this letter provided the basis for Marx's 1847 response, \textit{Misère de Philosophie}, a play on the subtitle of Proudhon's work, \textit{Philosophie de misère}.}: ``For him, the solution of present-day problems does not consist in public action but in the dialectical rotations of his brain.'' More bluntly, he accuses Proudhon of ``confus[ing] ideas and things,'' ``indulg[ing] in feeble Hegelianism in order to set himself up as an esprit fort,'' and thus tricking his audience via ``pseudo-Hegelian sleight-of-hand''. ``In a word, it is Hegelian trash,'' he concludes.

\cite{hoffman_revolutionary_1972} provides a much more charitable interpretation, positing essentially that although the ``de-Hegelianization'' of portions of Prouhdon's thought can be tedious, the benefits outweigh the costs. The tripartite Hegelian schema, Hoffman argues, provided a structure through which Proudhon was able to organize and communicate his ideas more straightforwardly than he otherwise would have, and (most relevant for the purposes of this work) made it easier for these ideas to travel across the continent. While the German socialist movement was rooted in Hegelian thought and rhetoric, and literate British socialists had been able to imbibe Hegelian ideas by way of Thomas Carlyle\footnote{}, the French socialist movement had been almost recalcitrant in their rejection of Hegelianism, as the Catholic socialists who predominated the movement were skeptical of its perceived atheism.

In the following plot, however, we see that indeed over the course of his entire adult life Proudhon oscillated between employing Hegelian rhetoric and concepts more so and less so, with no clear pattern of him ``coming down on'' one side or the other.

\begin{figure}[ht!]
    \centering
    \includesvg[width=\textwidth]{proudhon_pe_scores.svg}
    \caption{Proudhon's PE Scores over the course of his lifetime}
    \label{fig:proudhonpescores}
\end{figure}


\subsubsection{Marx vs. Lassalle\label{sec:marxvlassalle}}

Unlike in the case of Bakunin (discussed above, in Section \ref{sec:marxvauthors}), we do in fact have a large enough corpus of Lassalle's writings to perform a diachronic comparison of his and Marx's trajectory through ideological space across the span of their lives. We also give the caveat, however, that (as discussed in Section \ref{sec:lassalle}) much of Lassalle's time from the defeat of the 1848 Revolutions until the founding of the ADAV in 1863 was spent fighting a series of protracted legal battles: at first to secure his own release from prison, and then to ensure that the familial inheritance of his lifelong confidante Sophie von Hatzfeldt would not be usurped by other bitter rivals within her extended family.

Thus, as can be seen in Figure \ref{fig:lassallepubs}, we have a very limited amount of textual evidence from which to infer his ideological positions during the 1850s. We posit that the lack of data from this decade is not fatal, though, given that our interest in Lassalle's writings is only with respect to those of \textit{Marx}, whose correspondence with Lassalle (as seen in Figure \ref{fig:lassalleletters} began in earnest in 1856 and had mostly ended by 1860. With this in mind, we analyze Lassalle's trajectory not so much as a continuous path (like we did in the previous section) but rather with an eye towards whether or not a discontinuity is observed between his pre- and post-corresponding-with-Marx writings.

\begin{figure}[ht!]
    \centering
    \includesvg[width=\textwidth]{lassalle_pubs.svg}
    \caption{Lassalle's yearly literary output, 1840--1864.}
    \label{fig:lassallepubs}
\end{figure}

\begin{figure}[ht!]
    \centering
    \includesvg[width=\textwidth]{lassalle_letters.svg}
    \caption{The volume of correspondence between Marx and Lassalle, per year. Letters from Marx to Lassalle are tabulated based on \textit{MECW} Vols. 38--41. Letters from Lassalle to Marx are tabulated based on \textit{Ferdinand Lassalle. Nachgelassene Briefe und Schriften. Herausgegeben von Gustav Mayer}, Vol III: \textit{Der Briefwechsel zwischen Lassalle und Marx, nebst Briefen von Friedrich Engels und Jenny Marx an Lassalle und von Karl Marx an Gräfin Sophie Hatzfeldt}. (\cite{lassalle_ferdinand_1922}).}
    \label{fig:lassalleletters}
\end{figure}
