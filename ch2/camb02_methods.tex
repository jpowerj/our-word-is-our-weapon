\subsubsection{Author-Specific Embedding Spaces}

Although the embedding methods described in Section \ref{sec:methods} allowed us to construct a \textit{single} ideological space within which we were able to compare Marx's writings with those of his posited influences, in this section we need a more advanced technique which will allow us to trace the differential usage of various terms both over time and across authors (or groups of authors). Thus, for the explorations in this section we utilize a newer method introduced in
\cite{welch_exploring_2020}, %\cite{personalized_2018},
that of ``personalized'' word embeddings. While still estimating an overall ideological space based on the entire corpus (labeled \texttt{MAIN} in the resulting dataset), this approach also allows us to label each text with an author, for whom a separate set of embedding vectors is estimated.

Importantly, this method does not generate a separate embedding \textit{space} for every author, since one author's vectors need to be comparable with any other author's vectors\footnote{i.e., for authors $A$, $B$, and $C$, the distance $d(\vv{w_A}, \vv{w_B})$ between $A$'s vector for some word $w$ and $B$'s vector for $w$ must be on the same scale as the distance $d(\vv{w_A}, \vv{w_C})$ between $A$'s vector for $w$ and $C$'s vector for $w$, as well as the distance $d(\vv{w_B}, \vv{w_C})$ between $B$'s vector for $w$ and $C$'s vector for $w$.}. Instead, each author's specific vector $\vv{w_i}$ for a given term $w$ is estimated as some offset relative to the vector for $w$ in the \texttt{MAIN} vector space, $\vv{w_\texttt{MAIN}}$. Given a vector $\vv{w_{PE}}$ representing the centroid of political-economic discourse within the broader ideological vector space (estimated via a procedure we detail in the next section), for example, this allows us to instantly check whether an author $A$ tends to use a term $w$ in a more political-economic context than some other author $B$, by checking whether $d(\vv{w_A},\vv{w_{PE}}) < d(\vv{w_B},\vv{w_{PE}})$, or relative to the ``average'' usage of the term across the entire corpus, by checking whether $d(\vv{w_A},\vv{w_{PE}}) < d(\vv{w_\texttt{MAIN}},\vv{w_{PE}})$.

A problem arises, however, if we try to estimate an author-specific vector for an author with very few texts in the corpus, akin to e.g. the problem of statistical power in regression estimation. To address this issue, we instead group individual authors into ``meta-authors'' based on the discursive community they are generally associated with in the historical literature. Thus, for example, the texts of Bruno and Edgar Bauer, Arnold Ruge, Max Stirner, etc., are combined into one Young Hegelian meta-author in order to estimate a vector $\vv{w_{YH}}$ representing the centroid of Young Hegelian discourse (as defined by this author-to-group mapping) within the broader ideological space of 19th-century German discourse. Importantly, however, this approach is \textit{not} used to generate the political-economic and Hegelian vectors which serve as our orthogonal basis vectors, for reasons we describe in the next section.


\subsubsection{Discursive Fields as Embedding Clusters}

As mentioned in the previous section, there are two special vectors $\vv{w_{PE}}$ and $\vv{w_H}$, representing the centroids of political-economic and Hegelian discourse respectively, which we do \textit{not} compute via author-specific embedding estimation. Instead, to minimize the dependence (in the statistical sense) between our two basis vectors and the vectors like $\vv{w_{Marx}}$ for which we want to observe movement over time, we compute these basis vectors as centroids of word clusters which are derived independently via the cTFIDF measure, which generates a ranking of all terms in the corpus on the basis of how ``unique'' they are to political-economic texts relative to Hegelian texts (and vice-versa, by taking the $N$ terms with lowest, rather than greatest, cTFIDF scores).
