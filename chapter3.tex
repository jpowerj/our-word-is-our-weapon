%\title{Quantifying Cultural Diplomacy: The Translation and Diffusion of Marxism from the Communist Manifesto to the Cold War}
%\subtitle{Job Market Paper}
%\author{Jeff Jacobs\thanks{PhD Candidate, Political Science, Columbia University. (Job Market Paper, \href{https://polisci.columbia.edu/content/jeffrey-power-jacobs}{Click Here} for Job Market Profile)}\\\texttt{jpj2122@columbia.edu}}

\Chapter{Quantifying Cultural Diplomacy}{The Translation and Diffusion of Marxism from the Communist Manifesto to the Cold War}

\section*{Abstract}

We show, using a series of datasets compiled from declassified US and Soviet documents from 1945--1991, that the Soviet Union responded to political instability and regime change in Third World countries by massively increasing ``soft power'' projection into these countries. On average, a regime change (both CIA-backed and not) in a Third World country was associated with a \_\_\% increase in the volume of literature published in the languages of the country, and a \_\_\% increase in the number of scholarships granted to students in the country. We show that this finding holds from 1956, the beginning of Khrushchev's control over foreign policy decisions, until the end of the Soviet Union in 1991, regardless of the particular policymakers in charge of soft power decisions. We conclude with a comparative study of the exported literature, finding that ``overt'' propaganda discussing Marxism-Leninism and Communism in the USSR was most prevalent in industrialized Third World countries, while non-overt texts devoid of Communist doctrinal terminology were more prevalent in non-industrialized, agriculture-heavy countries.
%We conclude with a text-analytic study of the exported literature

\section{Introduction}

The purposeful export of culture, whether carried out by states and termed ``soft power'' in International Relations (Nye 2009) or carried out by non-state actors and termed ``internationalism'' in the study of revolutions (Halliday 1999), has long been recognized as a potent force for power projection in the global arena. Especially since Nye's seminal 1990 exposition of its role in global politics, interest in soft power has skyrocketed both within academia and more broadly in the eyes of state actors like Xi Jinping (Shambaugh 2015), international organizations like The Hague (Nye 2019), and corporate consulting firms like BrandFinance (Haigh 2020).

Despite its centrality to our contemporary understanding of international relations, and unlike the case for overt military, political, and economic coercion, \textit{empirical} studies of soft power projection and its effects remain few and far between. In this study, we leverage a series of datasets compiled from declassified US and Soviet documents to show that the Soviet Union disproportionately deployed its soft power capabilities in Third World countries in the immediate aftermath of regime changes in these countries. We argue that this reveals a strategy wherein the USSR aimed to capitalize on the uncertainty and instability of newly-installed regimes by flooding their populations with literary propaganda as well as massively increasing scholarships to study at the Patrice Lumumba People's Friendship University.

We also argue, in particular, that the USSR's propaganda exports were strategically targeted such that different ``Marxisms'' were constructed for different Third World regimes, each one drawing on a different subset of Marx's, Engels', and Lenin's writings. For example, the ideology exported to staunchly Non-Aligned post-colonial states like India focused more on children's education, indicating a concern with shaping the perception of the USSR in future generations, while in other post-colonial states like Ethiopia and Iran the imperative to seize state power was overtly emphasized, indicating the hopes of Soviet officialdom that their foothold among these countries' burgeoning urban elites could be leveraged to encourage revolution and the installation of a Soviet-friendly post-revolutionary regime. In a third case, for resistance movements in regions still fighting to overthrow colonial regimes like the Portuguese colonies in Southern Africa and Southeast Asia, the propaganda emphasized Marxism-Leninism as diametrically opposed to Western liberalism and portrayed the US as the torch-bearers of continued colonial dominance. We operationalize these hypotheses by examining the distribution of \textit{genres} exported to different regions at different times: how much weight was given to educational materials (e.g., textbooks), overtly Marxist-Leninist materials (e.g., the original writings and commentary on Marx, Engels, and Lenin), and more abstract philosophical or ethical works (e.g., historical works which emphasized the virtues of a region's anti-colonial heroes or martyrs such as Simon Bolivar in Latin America).

\section{Methods}

\section{Data}

\section{Results}

\section{Discussion}

\section{Conclusion}

